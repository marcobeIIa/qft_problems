\documentclass[a4paper]{article}

\usepackage[utf8]{inputenc}
\usepackage[T1]{fontenc}
\usepackage{textcomp}
\usepackage[english]{babel}
\usepackage{amsmath, amssymb, amsthm}
\usepackage{slashed}
\usepackage[a4paper, total={6in, 8in}]{geometry}
\usepackage{mathtools}
% figure support
\usepackage{import}
\usepackage{xifthen}
\pdfminorversion=7
\usepackage{pdfpages}
\usepackage{transparent}
\newcommand{\incfig}[1]{%
    \def\svgwidth{\columnwidth}
    \import{./figures/}{#1.pdf_tex}
}
\author{Marco Bella}
\title{Axions}
%evaluate symbol
\newcommand*\Eval[3]{\left.#1\right\rvert_{#2}^{#3}}
\DeclareMathOperator{\Tr}{Tr}
\pdfsuppresswarningpagegroup=1
\newtheorem*{exercise}{Exercise}

\begin{document}
\maketitle
\begin{exercise}[94.2, Srednicki]
Consider adding to the Standard Model a massless quark, represented
by a pair of Weyl fermions $\chi$ and $\xi$ in the $3$ and $\bar{3}$ representations
of SU(3). Also add a complex scalar $\Phi$ in the singlet representation. Assume that these fields have a Yukawa interaction of the form
$\mathcal{L}_\text{yuk} = y \Phi \chi \xi +\text{h.c.}$, where $y$ is the Yukawa coupling constant.
Assume that the scalar potential $V(\Phi)$ depends only on $\Phi^\dagger \Phi$.
\begin{enumerate}    \item Show that the lagrangian is invariant under a Peccei-Quinn
transformation $\chi \to  e^{i\alpha} \chi$, $\xi \to  e^{i\alpha}\xi$, $\Phi \to
e^{-2i\alpha}\Phi$, all other fields unchanged.
\item Show that this global $U(1)_\text{PQ} $ symmetry is anomalous, and that
$\theta\to \theta - 2\alpha$ under a $U(1)_\text{PQ} $ transformation.
\item Suppose that $V(\Phi)$ has its minimum at $\left|\Phi\right|= f / \sqrt{2} $, with
$f \neq 0$.
Show that this gives a mass to the quark we introduced.
\item Write $ \Phi = (f + \rho(x) ) e^{i a(x) /f} / \sqrt{2} $ , where $\rho$ and $a$ are fields. Argue that,
in eq. (94.10), we should replace $\theta$ with $\theta + a/f$, and add to eq. (94.9)
a kinetic term $+ \partial _\mu a \partial ^{\mu}a   $ for the $a$ field.
\item Show that the minimum of $V(U)$, defined in eq. (94.11), is at $U=I$ 
and $a = -f \theta$. Show that P and CP are conserved at this minimum.
\item The particle corresponding to the a field is the axion; compute its
mass, assuming $f \gg f_\pi$.
\item Note that if $f$ is large, the extra quark becomes very heavy, and
the axion becomes very light. Show that couplings of the axion to
the hadrons are all suppressed by a factor of $1 /f$. 
\end{enumerate}
\end{exercise}
\begin{enumerate}
   \item The new QCD lagrangian will be something like
   \begin{align}
       \mathcal{L}= -\frac{1}{4}F_{\mu\nu}^{a}F_{\mu\nu}^{a} + \bar{q}(i\slashed{D}-M)q +
       i \overline{\Psi} \slashed{D} \Psi + \left(\partial _\mu \Phi \right) ^\dagger 
      \left(\partial _{\mu}\Phi \right) - V(\Phi^\dagger \Phi) +\mathcal{L}_\theta+
      \mathcal{L}_\text{yuk} \label{lagrangian}
   \end{align}
(contracting low-low spacetime indices is abuse of notation but unambiguous) where 
   \begin{align*}
       q = \begin{bmatrix} u\\d \end{bmatrix} 
   \end{align*}
   contains 2 Dirac spinors $u,d$ (up and down quarks), 
   \begin{align*}
       M = \begin{bmatrix} m_u & 0 \\ 0 & m_d \end{bmatrix}
   \end{align*}
   is the quark mass matrix,  
   \begin{align*}
     \Psi = \begin{bmatrix} \chi \\ \xi ^\dagger  \end{bmatrix}  
   \end{align*}
   for two LH Weyls $\chi, \xi$ ($\xi^\dagger $ is then RH), and finally
   \begin{align*}
       \mathcal{L}_\theta = \frac{\overline{\theta} g^2}{64\pi^2}
       \varepsilon^{\mu\nu\alpha\beta}F_{\mu\nu}^{a}F_{\alpha\beta}^{a}
   \end{align*}
   is the CP-violating theta lagrangian.

   Under a PQ transformation, the new spinor becomes
   \begin{align}
   \Psi = \begin{bmatrix} \chi \\ \xi^\dagger  \end{bmatrix} \to \begin{bmatrix}
   e^{i\alpha}\chi \\ e^{-i\alpha}\xi^\dagger  \end{bmatrix} = e^{-i\gamma_5
   \alpha}\Psi\label{1}
   \end{align}
   where 
   \begin{align*}
       \gamma_5 = \begin{bmatrix} -1 & 0 \\ 0&1 \end{bmatrix} .
   \end{align*}
   Likewise,
   \begin{align*}
       \overline{\Psi} = \begin{bmatrix} \xi & \chi^\dagger  \end{bmatrix} \to
       \begin{bmatrix} e^{i\alpha}\xi & e^{-i\alpha}\chi^\dagger \end{bmatrix} =
       \overline{\Psi}e^{-i\gamma_5\alpha}.
   \end{align*}
   Thus the quark kinetic term is invariant (using $\{\gamma_5 , \gamma_\mu \} =0$):
   \begin{align*}
       i \overline{\psi}\slashed{D} \Psi =i \overline{\Psi} \gamma_\mu D^{\mu} \Psi &\to i \overline{\Psi}e^{-i\gamma_5
       \alpha} \gamma_\mu D^{\mu} e^{-i\gamma_5 \alpha} \Psi  \nonumber \\ 
       &= i \overline{\Psi} \gamma_\mu D^{\mu}
       e^{+i\gamma_5\alpha}e^{-i\gamma_5\alpha}\Psi = i \overline{\Psi}\slashed{D} \Psi;
   \end{align*}
  similarly the scalar kinetic term is trivially invariant
  \begin{align*}
      \partial _\mu \Phi^\dagger  \partial ^{\mu}\Phi &\to  \partial _\mu( e^{2i\alpha}
      \Phi ^\dagger)
       \partial _\mu  (e^{-2i\alpha} \Phi) \nonumber \\ & = \partial _\mu \Phi^\dagger
      \partial ^{\mu}\Phi  
  \end{align*}
  as is the Yukawa term:
  \begin{align*}
  \mathcal{L}_\text{yuk} =y \Phi \chi \xi +\text{h.c.} & \to  y (e^{-2i\alpha} \Phi )(e^{i\alpha}\chi) 
  (e^{i\alpha}\xi)+\text{h.c.} \nonumber \\ &= y \Phi \chi \xi + \text{h.c.}
  \end{align*}
  The potential $V(\Phi)$ by construction depends on $\Phi^\dagger \Phi$ which is
  obviously invariant. Every other term is unaltered by this tranformation. Thus the
  Lagrangian is invariant under this global PQ symmetry.
  \item As can be seen in \eqref{1}, the $U(1)_\text{PQ} $ is actually an axial $U(1)$. In
  section 76 it is shown that this type of symmetry is anomalous, in that the path
  integral integration measure picks up the following phase:
  \begin{align*}
  D \overline{\Psi}D \Psi \overset{U(1)_\text{PQ} }{\longrightarrow }   D \overline{\Psi} D\Psi \exp \left( -i\frac{g^2\alpha}{32
  \pi^2}\varepsilon^{\mu\nu\alpha\beta}F_{\mu\nu}^{a}F_{\alpha\beta}^{a} \right) 
  \end{align*}
  which effectively shifts the parameter $\overline{\theta}$ of $\mathcal{L}_\theta$:
  \begin{align*}
  \exp \left(-i\frac{\overline{\theta} g^2 }{ 64\pi^2}
  \varepsilon^{\mu\nu\alpha\beta}F_{\mu\nu}^{a}F_{\alpha\beta}^{a}\right)
  &\overset{U(1)_\text{PQ} }{\longrightarrow }    \exp
  \left(-i\frac{(\overline{\theta} -2\alpha)g^2 }{ 64\pi^2}
  \varepsilon^{\mu\nu\alpha\beta}F_{\mu\nu}^{a}F_{\alpha\beta}^{a}\right) \nonumber \\ 
      \overline{\theta} &\longrightarrow  \overline{\theta}-2\alpha .
  \end{align*}
  \item Working near the minimum of the potential to be at $\left|\Phi\right|= f /\sqrt{2}
  $, we can temporarily write 
  \begin{align*}
      \Phi = \frac{f}{\sqrt{2} }+ f(x)
  \end{align*}
  and the Yukawa coupling becomes
  \begin{align*}
      \mathcal{L}_\text{yuk} &= y \left(\frac{f}{\sqrt{2} }+f(x)\right)\chi \xi + y
      \left(\frac{f}{\sqrt{2} }+f^\dagger (x)\right)\xi^\dagger \chi^\dagger \nonumber \\ 
      &=\underbrace{ \frac{yf}{\sqrt{2} } \overline{\Psi} \Psi }_{\mathcal{L}_m}+
      \underbrace{(y^2 f(x) \chi \xi +\text{h.c.})
}_{\mathcal{L}^\prime _\text{yuk} }  \end{align*}
The mass lagrangian has a wrong sign, which can be fixed by a $U(1)_\text{PQ} $ with
$e^{i\alpha}=  i$. We then can read off the mass of the quark as 
\begin{align*}
    m_\Psi = \frac{yf}{\sqrt{2} }.
\end{align*}
\item By writing $\Phi = (f + \rho) e^{-ia /f } /\sqrt{2} $ we can immediately see that the
$a$ field enters as massless degree of freedom (NGB): the potential
$V(\left|\Phi\right|^2) = V(\rho)$ simply does not see the phase $a/ f$ of the field.
Neglecting the dynamics of $\rho$ and expanding around the minimum of $V$ ($\rho=0$), the kinetic term becomes
\begin{align*}
    \partial ^{\mu} \Phi^\dagger  \partial _\mu \Phi &= \frac{f^2}{2}  (\partial
    ^{\mu}e^{ia /f} ) (\partial _{\mu} e^{-ia /f})\nonumber \\ 
    &= \frac{1}{2}\partial ^{\mu}a \partial _\mu a  
\end{align*}
which is the kinetic term for a real scalar field. 

Under a $U(1)_\text{PQ} $ transformation, we have seen that
\begin{align*}
    \overline{\theta} \to  \overline{\theta} - 2\alpha
\end{align*}
due to the anomaly. Because of this, $\overline{\theta}$ by itself cannot have physical
relevance, as the global $U(1)_\text{PQ} $ can be seen as a simple relabelling of the
dummy integartion variables of the path integral, with regard to which physics should be agnostic. To restore physical meaning, we observe
that
\begin{align*}
    \text{Arg}\Phi &\to \text{Arg}\Phi -2\alpha\nonumber \\ 
    \frac{a}{f}&\to \frac{a}{f}+2\alpha
\end{align*}
so that the combination
\begin{align*}
    \overline{\theta} + \frac{a}{f} \to \overline{\theta} + \frac{a}{f}
\end{align*}
is invariant and becomes our new effective $\theta$ angle: we are free to
set $a = 0$ via axial rotations, but this will affect the theta lagrangian. Alternatively,
we could rotate away the theta lagrangian but this would shift the axion field around.
Both of these scenarios preserve the sum  $\overline{\theta}+ a /f$ which is our new effective
measure of CP violation. 

We are expanding around the minimum of $V$ because we are interested in the low-energy
theory. Let us temporarily remove axions from the discussion. The low-energy theory of QCD is described by the chiral
lagrangian, the details of which are beyond the scope of the exercise. A minimal
description is the following. If we neglect mass terms, the quark part of the lagrangian
\eqref{lagrangian} is
invariant under $SU(2)_\text{V} \times SU(2)_\text{A} $, among other things. The
$SU(2)_\text{V} $ part is realised as isospin symmetry, a consequence of which is e.g.
$m_p \approx m_n$. We have
good reasons to believe the axial part to be spontaneously broken instead. This results in 3 Goldstone
bosons, the pions, which can be modelled by a 2x2 matrix $U \in SU(2)$. We can then write
an effective field theory lagrangian with pions and nucleons as
degrees of freedom. For our purposes the pionic part of the lagrangian suffices, which looks like this:
\begin{align}
    \mathcal{L} = \frac{1}{4}f_\pi ^2 \Tr \partial ^{\mu}U^\dagger \partial _\mu U +v^3
    \Tr \left(MU + \text{h.c.}\right)  .\label{3}
\end{align}
There is a technical detail which requires some care. $M$ is still the quark mass
matrix, however it picks up the CP-violating phase that used to be in the theta lagrangian:
\begin{align*}
    M = \begin{bmatrix} m_u & 0 \\0 & m_d \end{bmatrix} e^{i \overline{\theta} /2}.
\end{align*}
The intuitive reason is that we did not include the theta term in our EFT lagrangian, but
our CP violation has to go somewhere. For our quark doublet $q = (\chi, \xi^\dagger )$, the mass lagrangian looks like
\begin{align*}
  \mathcal{L}_\text{mass} &= -\bar{q}Mq \nonumber \\ 
  &= - M_{ij}\chi_i \xi_j +\text{h.c.}
\end{align*}
Under a $U(1)_\text{A} $, 
\begin{align*}
    \mathcal{L}_\text{mass}  \to  -M_{ij}e^{2i\alpha} \chi_i \xi_j +\text{h.c.}
\end{align*}
and thus 
\begin{align*}
    M_{ij}\to M_{ij}e^{2i\alpha}.
\end{align*}
Since we are rotating two quarks now, and the theta lagrangian will pick up a $-2\alpha$ for
each anomalous rotation, the effective theta parameter undergoes
\begin{align*}
\mathcal{L}_\theta (\overline{\theta}) \to  \mathcal{L}_\theta
(\overline{\theta}-4\alpha).
\end{align*}
Then I can rotate the $\mathcal{L}_\theta$ out of the lagrangian, by setting $\alpha =
\overline{\theta} /4$, but I will pay half of that phase into the mass lagrangian:
\begin{align*}
\mathcal{L}_{\theta} &\to 0\nonumber \\ 
    M_{ij}&\to e^{i \overline{\theta} /2}
\end{align*}
as promised. When we put axions back into the treatment, the only thing we have to change
is 
\begin{align*}
    \overline{\theta} \xrightarrow[\text{axions}]{\text{include}}  \overline{\theta} +\frac{a}{f}.
\end{align*}
\item The previous point of the exercise gives us a potential for the low-energy theory
which depends on the axion field. The potential we can read off \eqref{3} as
\begin{align*}
    V(U,a) = - v^3 \Tr \left(MU+\text{h.c.}\right)
\end{align*}
with 
\begin{align*}
    M = \begin{bmatrix} m_u & 0 \\ 0 & m_d \end{bmatrix} e^{\frac{i}{2}(\overline{\theta}
    + a /f)}.
\end{align*}
For the minimum, we want to maximise the trace $\Tr MU$. As $U$ is an $SU(2)$ matrix, it
must be of the form
\begin{align*}
    U = \begin{bmatrix} c & b \\ -b^\star & c^\star\end{bmatrix} , \quad 
    \left|c\right|^2 +\left|b\right|^2 =1.
\end{align*}
The trace is
\begin{align*}
    \Tr MU = (m_u c + m_d c^\star )e^{\frac{i}{2}(\overline{\theta} + a /f)}
\end{align*}
and it is clear that our best shot at maximising it is for $b= 0$, because otherwise
$\left|c\right|^2 < 1$ (from the condition $\left|c\right|^2 + \left|b\right|^2 =1 $). In general, 
\begin{align*}
    U = \begin{bmatrix} e^{i\lambda} &0 \\ 0 & e^{-i\lambda} \end{bmatrix} 
\end{align*}
and the potential becomes
\begin{align*}
    V(\lambda, a) &= -v^3 \left[m_ue^{i\lambda + \frac{1}{2}(\overline{\theta}+ a /f)} +
    m_de^{-i\lambda + \frac{1}{2}(\overline{\theta}+ a /f)}\right]\nonumber \\ 
    &= -2v^3 \left\{m_u\cos\left[\lambda +\frac{1}{2}\left(\overline{\theta} +
    \frac{a}{f}\right)\right] + m_d \cos\left[\lambda -\frac{1}{2}\left(\overline{\theta}
    + \frac{a}{f}\right)\right]\right\}
\end{align*}
which is minimised for $\cos\left(\ldots\right)=1$ (for both cosines), realised for
\begin{align*}
    \lambda &= 0  \quad \implies U = \mathbb{I}\nonumber \\ 
    \overline{\theta}+ \frac{a}{f}&=0
\end{align*}
i.e. the CP-violating term relaxes to zero. 

\item The potential we have just written explicitly breaks the $U(1)_\text{PQ} $ symmetry
by giving the axion a potential. Axion mass can be computed expanding the axion field as
\begin{align*}
    a = -f \overline{\theta} + \tilde{a},
\end{align*}
and then calculating the curvature in the $\tilde{a}$ direction. Here we exploit the assumption $f\gg f_\pi$
of the exercise, which means the following: minimise the energy with respect to $\lambda$,
and then expand the potential which now only depends on $\tilde{a}$. The assumption is
equivalent to assuming that pions have infinite mass ($m_\pi \sim  1 /f_\pi$ it turns out), so that they decouple from the
dynamics and we simply set their potential to be at the minimum:
\begin{align*}
    \frac{\partial V}{\partial \lambda} &= 2v^3 \left\{m_u \sin \left[\lambda +
    \frac{\tilde{a}}{2f}\right] + m_d \sin \left[\lambda -
    \frac{\tilde{a}}{2f}\right]\right\}\nonumber \\ 
    & = 2v^3\left[m_u \left(\sin\lambda \cos \frac{\tilde{a}}{2f} +\cos\lambda \sin
    \frac{\tilde{a}}{2f}\right) + m_d \left(\sin\lambda \cos \frac{\tilde{a}}{2f} -
    \cos\lambda \sin \frac{\tilde{a}}{2f}\right) \right] \overset{!}{=} 0
\end{align*}
and the minimum is thus at
\begin{align*}
(m_u+m_d)\sin \lambda \cos \frac{\tilde{a}}{2f}  = (m_d-m_u) \cos\lambda \sin
\frac{\tilde{a}}{2f}
\end{align*}
or 
\begin{align*}
    \tan \lambda &= \frac{m_d-m_u}{m_d + m_u} \tan \frac{\tilde{a}}{2f}\nonumber \\ 
    \lambda &= \frac{m_d - m_u}{m_d + m_u} \tan \frac{\tilde{a}}{2f} +
    \mathcal{O}\left(\frac{\tilde{a}}{2f}\right)^3\nonumber \\ 
    &=  \frac{m_d - m_u}{m_d + m_u}\frac{\tilde{a}}{2f}+
    \mathcal{O}\left(\frac{\tilde{a}}{2f}\right)^3.
\end{align*}
Now we removed $\lambda$ from our potential, and after some manipulations we shall see the
mass term for the axion:
\begin{align*}
    V(\tilde{a}) &= -2v^3 \left\{m_u \cos\left[\lambda + \frac{\tilde{a}}{2f}\right] + m_d
    \cos\left[\lambda - \frac{\tilde{a}}{2f}\right]\right\}\nonumber \\ 
    &= -2v^3 \left\{m_u \cos \left[\left(\frac{m_d-m_u}{m_d+m_u} +1\right)
    \frac{\tilde{a}}{2f}\right]  + m_d \cos \left[\left(\frac{m_d-m_u}{m_d+m_u}
    -1\right) \frac{ \tilde{a}}{2f}\right] \right\}\nonumber \\ 
    &= -2v^3 \left\{m_u \cos \left[\frac{m_d}{m_d+m_u}
    \frac{\tilde{a}}{f}\right]  + m_d \cos \left[\frac{m_u}{m_d+m_u}
    \frac{\tilde{a}}{f}\right]\right\},\nonumber \\ 
    \intertext{Taylor expanding $\cos x \approx 1- x^2 /2 $:} 
    &\approx v^3 \left\{m_u \left[\frac{m_d}{m_d+m_u} \frac{\tilde{a}}{f}\right]^2 + m_d
    \left[\frac{m_u}{m_d+m_u} \frac{\tilde{a}}{f}\right]^2\right\} + \text{const}\nonumber \\ 
    &= v^3 \frac{m_um_d}{m_u+m_d} \frac{\tilde{a}^2}{f^2} + \text{const}
\end{align*}
and we can read off the mass as $V \sim  m^2 \tilde{a}^2 /2$,
\begin{align*}
    m_a^2 = \frac{2v^3 m_um_d}{(m_u+m_d)f^2}.
\end{align*}
\item We have just found that $m_a \sim  1/f$, whereas in the 3rd point we showed $m_\Psi
\sim  f$. Thus as $f$ increases, so will the mass of the new quark, whereas the axion mass
will decrease. In our lagrangian we never actually see the axion field by itself, as it
enters as a phase $a /f$. Thus any possible interaction coming out of this low-energy
lagrangian will be proportional to  $1 /f$, which we can interpret as a coupling constant.
Any coupling will carry this suppression (it is a suppression as $[f] = 1 $ today is constrained to  $f>10^{10}$
GeV roughly).
\end{enumerate}
 
\end{document}
