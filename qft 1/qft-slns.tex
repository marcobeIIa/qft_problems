% !TeX program = luatex
% Finished mode.
\documentclass[a4paper]{article}
\usepackage[usenames,dvipsnames,table]{xcolor}
\usepackage{enumitem}
\usepackage{flowfram}
\usepackage{lipsum}
\usepackage[T1]{fontenc}
\usepackage{tikz}
\usepackage{pifont}
%\usepackage{mathabx}
\usepackage{hhline}
\usepackage[nice]{nicefrac}
\usepackage{booktabs}
\usepackage{colortbl}
\usepackage{tikz-feynman}
\usepackage{simpler-wick}
%% cose cursed
\pgfmathdeclarerandomlist{MyRandomColors}{%
    {red}%
    {blue}%
    {cyan}%
    {orange}%
    {yellow}%
}%
%what
\tikzset{rubout/.style={preaction={draw=white,line width=3pt}}}
%%% fine cose cursed
\usepackage[utf8]{inputenc}
\usepackage[T1]{fontenc}
\usepackage{textcomp}
\usepackage[english]{babel}
\usepackage{amsmath, amssymb,amsthm}
\usepackage{mathtools}
\usepackage{tensor}
\usepackage{cancel}
\usepackage{siunitx}
% figure support
\usepackage{transparent}
\usepackage{slashed}
%evaluate symbol
\newcommand*\Eval[3]{\left.#1\right\rvert_{#2}^{#3}}
\newcommand{\incfig}[1]{%
    \def\svgwidth{\columnwidth}
    \import{./figures/}{#1.pdf_tex}
}

\usepackage[a4paper, total={16cm, 24cm}]{geometry}
\usepackage{amsmath,thmtools}
\usepackage{physics}
\usepackage{textgreek}
\usepackage[most]{tcolorbox}
%\title{QMFS Exercises}
\title{QFT final problems}
\date{\today}
\author{Marco Bella}

\DeclareSIUnit\ev{eV}

\declaretheoremstyle[
  headfont=\color{Bittersweet}\normalfont\bfseries,
  bodyfont=\color{NavyBlue}\normalfont\itshape,
]{colored}
\declaretheorem[
  name=Remark,
]{rmk}
\declaretheorem[
  name=Proposition,
]{prop}

\declaretheorem[
  name=Definition,
]{definition}

\declaretheorem[
  name=Lemma,
]{lemma}

\declaretheorem[
  style=colored,
  name=Exercise,
]{exercise}
\begin{document}
\maketitle
\section{Exercise 1}%
    \begin{exercise}
        When quantizing the electromagnetic field in a covariant form, we used a Lagrangian of this form:
        \begin{align}
        \mathcal{L}_0= -\frac{1}{2}\partial ^{\nu}A^\mu \, \partial _{\nu}A_\mu  
        \end{align}
Show that the alternative Lagrangian:
\begin{align}
\mathcal{L} = -\frac{1}{4}F^{\nu\mu}F_{\nu\mu}- \frac{1}{2} (\partial _\mu A^{\mu}
)^2  \label{1.2}
\end{align}
would have produced the same quantization.
    \end{exercise}
    This is seen most simply by showing that the Lagrangians produce the same action.
    Explicitly, the Lagrangian \eqref{1.2} can be expanded as\footnote{using comma
    notation for derivatives ($\partial _\mu f^{\nu} = \tensor[]{f}{^{\nu}_{,\mu}} $) and
    introducing antisymmetric brackets as $f_{[\mu\nu]}= f_{\mu\nu} - f_{\nu\mu}$.}
    \begin{align}
        \mathcal{L} &= -\frac{1}{4} A_{\left[\mu,\nu\right]}A^{\left[\mu,\nu\right]} -
        \frac{1}{2} \left(\tensor[]{A}{^{\rho}_{,\rho}}\right)^2\nonumber \\ &=
-      \frac{1}{4}  \left(A_{\mu,\nu}- A_{\nu,\mu}\right)\left(A^{\mu,\nu} - A^{\nu,\mu}\right) -
        \frac{1}{2} \left(\tensor[]{A}{^{\rho}_{,\rho}}\right)^2\nonumber \\ 
        &= -\frac{1}{4}\left(A_{\mu,\nu}A^{\mu,\nu} - A_{\mu,\nu}A^{\nu,\mu} -
        A_{\nu,\mu}A^{\mu,\nu} + A_{\nu,\mu}A^{\nu,\mu}\right) -
        \frac{1}{2}\left(\tensor[]{A}{^{\rho}_{,\rho}}\right)^2.
    \end{align}
    Relabelling dummy indices yields
    \begin{align}
        \mathcal{L} = -\frac{1}{2} A_{\mu,\nu}A^{\mu,\nu} + \frac{1}{2} A_{\mu,\nu}
        A^{\nu,\mu} - \frac{1}{2} \left(\tensor[]{A}{^{\rho}_{,\rho}}\right)^2. \label{1.3}
    \end{align}
    We can now integrate by parts twice on the second term:
    \begin{align}
        \mathcal{L} &= -\frac{1}{2} A_{\mu,\nu}A^{\mu,\nu} - \frac{1}{2} A_{\mu}
               \tensor[]{A}{^{\nu}^{,\mu}_{\nu}} - \frac{1}{2} \left(\tensor[]{A}{^{\rho}_{,\rho}}\right)^2 \nonumber \\ 
        &= -\frac{1}{2} A_{\mu,\nu}A^{\mu,\nu} - \frac{1}{2} A_{\mu}
        \tensor[]{A}{^{\nu}_{,\nu}^{\mu}} - \frac{1}{2} \left(\tensor[]{A}{^{\rho}_{,\rho}}\right)^2 \nonumber \\ 
        &= -\frac{1}{2} A_{\mu,\nu}A^{\mu,\nu} + \frac{1}{2} \tensor[]{A}{_\mu^{,\mu}}
        \tensor[]{A}{^{\nu}_{,\nu}}- \frac{1}{2}
        \left(\tensor[]{A}{^{\rho}_{,\rho}}\right)^2 , 
    \end{align}
    where in the second line I employed Schwarz's theorem, commuting partial derivatives,
    and all the equalities are understood in terms of the action (assuming boundary terms
    to vanish upon integration, as is customary in field theories).
    The second and third term in the last line cancel out and we retrieve the Lagrangian
    $\mathcal{L}_0$.

    If the Lagrangians are equivalent, the theories coincide even at a classical level. 

  %  I
  %  still went and looked at what happens when we try to quantise the theory, for the spirit of the
  %  exercise.

%    This
%    Lagrangian calls for the same quantisation procedure as the classroom case, namely
%    the Gupta-Bleuler approach.
%
%    We can start
%        off by computing the Euler-Lagrange equations. Starting from \eqref{1.3},
%    \begin{align}
% \frac{\partial \mathcal{L}}{\partial \left(A_{\mu,\nu}\right)} &= -\frac{1}{2} \frac{\partial
% }{\partial\left(A_{\mu,\nu}\right)}\left(A_{\rho,\sigma}A^{\rho,\sigma} - A_{\sigma,\rho}A^{\rho,\sigma} +
%\tensor[]{A}{^{\rho}_{,\rho}}\tensor[]{A}{^{\sigma}_{,\sigma}}\right)\nonumber \\ 
%&= -2 \frac{1}{2} \left(\delta_{\rho}^{\mu}\delta_{\sigma}^{\nu}A^{\rho,\sigma} -
%\delta_{\sigma}^{\mu}\delta_{\rho}^{\nu}A^{\rho,\sigma} +
%\eta^{\rho\mu}\delta_{\rho}^{\nu}\tensor[]{A}{^{\sigma}_{,\sigma}}\right)\nonumber \\ 
%&= -A^{\mu,\nu} + A^{\nu,\mu} -
%\eta^{\mu\nu}\tensor[]{A}{^{\sigma}_{,\sigma}}. \label{1.6}
%    \end{align}
%    Differentiating:
%    \begin{align}
%0&\overset{\text{(EL)}}{=}   \partial _\nu \left(\frac{\partial \mathcal{L}}{\partial \left(A_{\mu,\nu}\right)}
%        \right)= - \tensor[]{A}{^{\mu,\nu}_{\nu}}  + \tensor[]{A}{^{\nu,\mu}_{\nu}} -
%        \eta^{\mu\nu}\tensor[]{A}{^{\sigma}_{,\sigma}_{\nu}}.
%    \end{align}
%    Manipulating the last term ($\eta^{\mu\nu}\tensor[]{A}{^{\sigma}_{,\sigma\nu}} =
%    \tensor[]{A}{^{\sigma}_{,\sigma}^{\mu}} = \tensor[]{A}{^{\nu}_{,\nu}^{\mu}} =
%    \tensor[]{A}{^{\nu}^{,\mu}_{\nu}}$) we see that it cancels with the second one. Thus the
%    equations of motion reduce to
%    \begin{align}
%            \square A^{\mu}=0.
%    \end{align}
%    This is the usual set of KG equations, whose solution was discussed thoroughly in
%    class (hence I will just skim through this bit). In particular one goes to momentum
%    space, where the equations reduce to the
%    (massless) particle on-shell relation, $p^2 = 0$, and goes back to coordinate space
%    via plane wave superposition. Since our $A^{\mu}$ is a 4-vector, we will decompose it
%    on a basis of polarisation 4-vectors $\varepsilon_\lambda^{\mu}(p)$:
%    \begin{align}
%        A^{\mu}= \int \frac{\mathop{d^3p} }{(2\pi)^3}\frac{1}{\sqrt{2\omega_p} }
%        \sum_{\lambda=0}^{3} \left( \varepsilon_\lambda^{\mu}(p) a_\lambda(p) e^{-i p\cdot x} +
%        \text{h.c.}
% \right) .    \end{align}
% Quantisation follows as in class: we promote $\mathcal{L}, A^{\mu},a_\lambda(p)$ to
% operators (introducing normal ordering as usual in the Lagrangian), and postulate
% commutator relations between the fields and the conjugate momenta.
%
% In particular, the conjugate momenta are (from \eqref{1.6})
% \begin{align}
%     \Pi^{\mu}= \frac{\partial \mathcal{L}}{\partial \left(A_{\mu,0}\right)} = -A^{\mu,0}
%     + A^{0,\mu}- \eta^{\mu 0 }\tensor[]{A}{^{\sigma}_{,\sigma}},
% \end{align}
% or
% \begin{align}
%     \Pi^{i}&= -F^{0i} \nonumber \\ 
%     \Pi^{0}&= -\tensor[]{A}{^{\sigma}_{,\sigma}}.
% \end{align}
% All the conjugate momenta are nonzero. I impose equal-time commutator
% relations between the field operators:
% \begin{align}
%     \left[A^{\mu}(x),\Pi^{\nu}(y)\right]_{x^{0}=y^{0}} &= i \eta^{\mu\nu} \delta^{(3)}
%     (\textbf{x}-\textbf{y}),\nonumber \\ 
%     \left[A^{\mu}(x),A^{\nu}(y)\right]_{x^{0}=y^{0}} &=
%     \left[\Pi^{\mu}(x),\Pi^{\nu}(y)\right]_{x^{0}=y^{0}} =0 .
%     \end{align}
%     The next step is to show that these relations impose a certain algebra on the
%     operators $a_\lambda (p), a_{\lambda^\prime}(p^\prime)$. I do this by reduction to a
%     known problem, as inverting the explicit form of the conjugate momenta is cumbersome. In
%     particular, one can see that 
%     \begin{align}
%      \left[A^{\mu}(x),\Pi^{\nu}(y)\right]_{x^{0}=y^{0}} &= i \eta^{\mu\nu} \delta^{(3)}
%     (\textbf{x}-\textbf{y})       \implies
%     \left[A^{\mu}(x),\dot{A}^{\nu}(y)\right]_{x^{0}=y^{0}} =- i \eta^{\mu\nu}
%     \delta^{(3)}  (\textbf{x}-\textbf{y})\label{1.13}
%     \end{align}
%     Once that is cleared, one might as well forget about conjugate fields; the commutator
%     relations for the ladder operators follow by virtue of the same calculation
%     performed in class:
%     \begin{align}
%        \left[a_\lambda (p) , a^\dagger _{\lambda^\prime }(p^\prime )\right] = (2\pi)^3
%        \delta^{(3)} (\textbf{p}- \textbf{p}^\prime)\delta_{\lambda \lambda^\prime
%        }\alpha_\lambda.
%     \end{align}
%         The point behind \eqref{1.13} is that terms involving spatial derivatives vanish, due to the fact the
%         field commutes with itself at equal times: terms vanish of the type
%         \begin{align}
%            \left[A^{\mu},\tensor[]{A}{^{\nu}_{,i}}\right]_{x^{0}= y^{0}}= \lim_{\varepsilon \to 0}
%            \frac{\left[A^{\mu}(x^{0},\textbf{x}), A^{\nu}(x^{0}, \textbf{y}+\varepsilon
%            \hat{\textbf{e}}_i)\right] - \left[A^{\mu}(x^{0},\textbf{x}),
%            A^{\nu}(x^{0},\textbf{y})\right]}{\varepsilon} = \lim_{\varepsilon \to 0}
%            \frac{0-0}{\varepsilon} = 0.
%         \end{align}
%         Thus, if $\nu=0$, 
% \begin{align}
%     [A^{\mu}(x), \Pi^{0}(y) ]_{x^{0}= y^{0}} &= - \left[A^{\mu}(x), \tensor[]{A}{^{0}_{,0}}(y)\right] +
%     \left[A^{\mu}(x),\tensor[]{A}{^{i}_{,i}}(y)\right] \nonumber \\ 
%     &= -  \left[A^{\mu}(x), \tensor[]{\dot{A}}{^{0}}(y)\right];
% \end{align}
% if $\nu = i $,
% \begin{align}
%     \left[A^{\mu}(x), A^{i}(y)\right]_{x^{0}= y^{0}} &= -\left[A^{\mu}(x),
%     A^{i,0}(y)\right]+ \left[A^{\mu}(x), A^{0,i}(y)\right] \nonumber \\ 
%     &= - \left[A^{\mu}(x), \dot{A}^{i}(y)\right].
% \end{align}
% Thus, for all $\nu$, $[A^{\mu}(x),\dot{A}^{\nu}(y) ] = [A^{\mu}(x),\Pi^{\nu}(y)] = -i
% \eta^{\mu\nu}\delta^{3}(\textbf{x}-\textbf{y})$. This is the relation we worked with in
% class. 
%
% From here onwards, all that matters is the field commutators, which lead
% to the ladder operator algebra, and the equations of motion to restrict the Fock space of
%physical states. We have just proven the two Lagrangian densities to produce the same
%equations of motion and the same algebra. Thus they produce the same quantisation.
\section{Exercise 2}%

 \begin{exercise}
    Consider two scalar, complex free fields, $\phi_{a,b}(x)$, with the same mass $m$, subject
to the following action:
\begin{align}
\mathcal{S} = \int \mathop{d^4x}\left(\partial _\mu \phi_a ^{\star} \partial
^{\mu}\phi_a +  \partial _\mu \phi_b ^{\star} \partial
^{\mu}\phi_b +  m^2 \left(\phi_a^{\star}\phi_a + \phi_b^{\star}\phi_b\right) \right) 
\end{align}
\begin{itemize}
    \item Show that the above action has 4 conserved charges. Define $Q$ as the
    charge that is the obvious generalization of the charge of a single complex
    field, and define the other three charges as $Q^{i}$. 
    \item Find the commutation relations between the charges $Q^{i}$. What is the
    underlying symmetry group? 
\end{itemize}
 \end{exercise}
 For the sake of compactness, I will temporarily group the fields into 2-vectors (not
 actually vectors, abuse of notation):
 \begin{align}
 \psi \coloneqq \begin{bmatrix} \phi_a \\ \phi_b \end{bmatrix} , \quad \psi^\dagger =
\begin{bmatrix}  \phi_a^\star & \phi_b^\star \end{bmatrix}.
 \end{align}
 The Lagrangian density becomes
 \begin{align}
     \mathcal{L} = \partial _\mu \psi^\dagger  \partial ^\mu \psi + m^2 \psi^\dagger \psi
     .
 \end{align}
 This Lagrangian is invariant under a global\footnote{In the sense
 that $U$ is just a constant matrix, $\partial_\mu U = 0 $.} unitary transformation of the field 2-vector:
 \begin{align}
     \psi &\to U \psi, \nonumber \\ 
     \psi^\dagger  &\to  \psi^\dagger U^\dagger \nonumber \\ 
     \mathcal{L} &\to  \partial _\mu \psi^\dagger U^\dagger U \partial ^{\mu} \psi + m^2
     \psi^\dagger U^\dagger U \psi  \nonumber \\ 
     &=  \partial _\mu \psi^\dagger  \partial ^\mu \psi + m^2 \psi^\dagger \psi =
     \mathcal{L}.
 \end{align}
The group of 2x2 unitary matrices is $U(2)$. The algebra of $U(2)$ is the vector space of 2x2
hermitian matrices, as seen by writing $U(2) \ni M = \exp (i \varepsilon_\mu
s^{\mu})$ for some matrices $s^{\mu}$:\footnote{This can always be done as unitary matrices are normal and thus
diagonalisable. In diagonal form the matrix exponential is easily manipulated.} $M^\dagger = \exp (-i \varepsilon_\mu (s^\dagger )^{\mu})$. At any order in
the matrix exponential, it must hold that $(s^\dagger )^{\mu}= s^{\mu}$. How many
hermitian 2x2 matrices are there? We are free to pick two real numbers on the diagonal, and one complex number for the off-diagonal entries. So there are four. The basis choice
falls naturally on 
\begin{align}
    s^{\mu}= \left(\mathbb{I}, \frac{ \boldsymbol{\sigma}}{2}\right).
\end{align}
    We can write an infinitesimal $U(2)$ transformation in terms of these generators:
    \begin{align}
        \Psi \to  \left(\mathbb{I} + i \varepsilon_\mu s^{\mu}\right)\Psi = \begin{cases}
            \psi + i \varepsilon_0 \psi = \psi + \varepsilon_0 \delta \psi^{0} \\
            \psi + \frac{i}{2} \varepsilon_j \sigma^{j} \psi = \psi + \varepsilon_j \delta
            \psi^{j}.
        \end{cases}
    \end{align}
The Noether currents take the form
 \begin{align}
    (j^{\mu})_\alpha &= \frac{\partial \mathcal{L}}{\partial \left(\partial ^{\alpha} \psi \right)}
    \delta \psi ^{\mu} + \left(\delta \psi^{\mu} \right)^\dagger  \frac{\partial
    \mathcal{L}}{\partial \left(\partial ^{\alpha} \psi^\dagger  \right)} \nonumber \\ 
    &= \partial _\alpha \psi^\dagger \delta \psi^{\mu} + \left( \delta\psi^\dagger
    \right)^{\mu} \partial _\alpha \psi \nonumber \\ 
    &= \partial _\alpha \psi^\dagger  (i s^{\mu}) \psi + \text{h.c.} 
\end{align}
The conserved charges are 
\begin{align}
    Q &= \int \mathop{d^3x} (j^{0})_0 \nonumber \\ 
    &= i \int \mathop{d^3x} \left( \partial _0 \psi^\dagger \psi - \psi^\dagger \partial
    _0 \psi  \right) \nonumber \\ 
    &= i \int \mathop{d^3x} \left(\dot{\psi}^\dagger \psi - \psi^\dagger \dot{\psi}\right) 
\end{align}
and, likewise,
\begin{align}
    Q^{i} &= \int \mathop{d^3x} (j^{i})_0 \nonumber \\ 
    &= \frac{i}{2}\int \mathop{d^3x} \left( \dot{\psi}^\dagger  \sigma^{i}\psi -
    \psi^\dagger \sigma^{i} \dot{\psi}\right)  .
\end{align}
As is customary, we can express the charge operator in terms of ladder operators. Here I
found it favourable to bring back the scalar fields. 
\begin{align}
    Q^{\mu} = i\int \mathop{d^3x} \left(\dot{\phi}^\star_j (s^{\mu})^{jk}\phi_k
    - \text{h.c.}\right) 
\end{align}
where Einstein summation is implicit on repeated high-low indices.
The fields are
\begin{align}
    \phi_j &=\int \frac{\mathop{d^3p}}{(2\pi)^3}\frac{1}{\sqrt{2E_{p}}}
    \left(a_{\textbf{p}j} e^{-ip\cdot x} + b_{\textbf{p} j }^\dagger e^{ip\cdot
    x}\right)_{p^{0}= E_p},\nonumber \\ 
   \dot{ \phi}_j &= \int
   \frac{\mathop{d^3p}}{(2\pi)^3}\frac{1}{\sqrt{2E_{p}}}\left(a_{\textbf{p}j}(-i E_p)
   e^{ip\cdot x} + b^\dagger _{\textbf{p}j}\left(i E_p\right)e^{ip\cdot x}\right)\nonumber \\ 
   &= -i\int \frac{\mathop{d^3p}}{(2\pi)^3}\sqrt{\frac{E_p}{2}} 
    \left(a_{\textbf{p}j} e^{-ip\cdot x} - b_{\textbf{p} j }^\dagger e^{ip\cdot
    x}\right)_{p^{0}= E_p}.
\end{align}
Overall, the charges are 
\begin{align}
    Q^{\mu}&= i\int \mathop{d^3x}
    \left(\left(s^{\mu}\right)^{jk}\dot{\phi}^\star_j \phi_k - \text{h.c.} \right)\nonumber \\ 
    & = i \int \mathop{d^3x}  (+i)\int
    \frac{\mathop{d^3p}}{(2\pi)^3}\sqrt{\frac{E_p}{2}}\int
    \frac{\mathop{d^3q}}{(2\pi)^3}\frac{1}{\sqrt{2E_{q}}} \times \nonumber \\  
    & \phantom{=} \times \left\{ \left(s^{\mu}\right)^{jk}\left[\left(a_{\textbf{p}j}^\dagger e^{ip\cdot x} -
    b_{\textbf{p}j}e^{-ip\cdot x} \right)\left(a_{\textbf{q}k}e^{-iq\cdot x} +
    b_{\textbf{q}k}^\dagger e^{iq\cdot x}\right)\right] + \text{h.c.}\right\} 
    \end{align}
    where the $- \text{h.c.}$ became a $+\text{h.c.}$ as I grouped a $+i$ outside. The integrals in $\mathop{d^3x} $ of the plane waves give $(2\pi)^3
    \delta^{(3)}(\textbf{p} \pm \textbf{q})$. These deltas kill off the factor $\sqrt{E_p / E_q} $ as they
    force $\left|\textbf{p}\right|= \left|\textbf{q}\right|$.
    \begin{align}
        Q^{\mu} &= - \int
        \frac{\mathop{d^3p}}{(2\pi)^3}\frac{1}{2} \left\{\left( s^{\mu} \right)^{jk}\left[ a_{\textbf{p}j}^\dagger
        a_{\textbf{p}k}+ a_{\textbf{p}j}^\dagger b_{-\textbf{p}k}^\dagger -
        b_{\textbf{p}j}a_{-\textbf{p}k}- b_{\textbf{p}j}b_{\textbf{p}k}^\dagger
       \right] +\text{h.c.} \right\}. 
    \end{align}
    The Hermitian conjugate of the first and last term gives back itself, due to the
    hermiticity of the generators $s^{\mu}$: e.g. $( a^\dagger
    _{\textbf{p}j}\left(s^{\mu}\right)^{jk}a_{\textbf{p}k} ) ^\dagger  =
    a_{\textbf{p}k}^\dagger(s^{\mu})^{kj}a_{\textbf{p}j}$. This cancels the factor $1 /2$.
    The Hermitian conjugate of the second and third term cross-cancel: e.g. the second
    term becomes
     \begin{align}
        \left(a_{\textbf{p}j}^\dagger (s^{\mu})^{jk} b_{-\textbf{p}k}^\dagger
        \right)^\dagger  = b_{-\textbf{p}k}(s^{\mu})^{kj} a_{\textbf{p}j}
    \end{align}
    which, upon substituting $\textbf{p}\to -\textbf{p}$, kills the third term. The same
    fate awaits the other two terms; thus (dropping the overall minus sign for aesthetics)
    \begin{align}
        Q^{\mu} = \int \frac{\mathop{d^3p} }{(2\pi)^3} \left(a^\dagger
        _{\textbf{p}j}\left(s^{\mu}\right)^{jk}a_{\textbf{p}k} - b^\dagger
        _{\textbf{p}j}\left(s^{\mu}\right)^{jk}b_{\textbf{p}j}\right),
    \end{align}
    or
    \begin{align}
        \boxed{
            Q = \int \frac{\mathop{d^3p} }{(2\pi)^3} \left(a^\dagger
            _{\textbf{p}j}a_{\textbf{p}}^{j} - b^\dagger _{\textbf{p}j} b_{\textbf{p}}^{j}
            \right)
            }
        \end{align}
        and
        \begin{align}
        \boxed{
            Q^{i}= \frac{1}{2}\int \frac{\mathop{d^3p} }{(2\pi)^3} \left(a^\dagger
            _{\textbf{p}j}\left( \sigma^{i} \right) ^{jk}a_{\textbf{p}k} - b^\dagger
            _{\textbf{p}j}\left( \sigma^{i} \right) ^{jk}
            b_{\textbf{p}k} \right).
            }
    \end{align}
    \subsection{Commutator relations}%
    
    Now onto the commutator relations. When commuting two charges, we will have to work
    with commutators of the type
    \begin{align}
        \left[AB,CD\right] &= A\left[B,CD\right] + \left[A,CD\right]B \nonumber \\ 
        &= AC\left[B,D\right] + A\left[B,C\right]D + \left[A,C\right]DB +
        C\left[A,D\right]B, \label{2.35}
    \end{align}
    where $A, C$ are creation operators and  $B,D$ are annihilation operators. Thus the only
    nonvanishing combinations in \eqref{2.35} are those commuting $A,D$ and  $B,C$.
    \begin{align}
        \left[Q^{\mu},Q^{\nu}\right] &= \int \frac{\mathop{d^3p}\mathop{d^3q}
        }{(2\pi)^{6}}\left[a_{\textbf{p}i}^\dagger (s^{\mu})^{ij}a_{\textbf{p}j} -
        \left(a\leftrightarrow b\right),a_{\textbf{q}k}^\dagger (s^{\nu})^{kl}a_{\textbf{q}l} -
\left(   a\leftrightarrow b\right) \right]\nonumber \\ 
&= \int \frac{\mathop{d^3p}\mathop{d^3q}  }{(2\pi)^{6}} a_{\textbf{p}i}^\dagger
(\sigma^{\mu})^{ij} \left[a_{\textbf{p}j}, a_{\textbf{q}k}^\dagger \right](\sigma^{\nu})
^{kl}a_{\textbf{q}l} + a^\dagger _{\textbf{q}k} (s^{\nu})^{kl}\left[a^\dagger
_{\textbf{p}i}, a_{\textbf{q}l}\right](s ^{\mu})^{ij}a_{\textbf{p} j} +
\left(a\leftrightarrow b\right) .
    \end{align}
The ladder operator algebra for two free complex scalar fields is
\begin{align}
    \left[a_{\textbf{p}j}, a_{\textbf{q}k}^\dagger \right] = (2\pi)^3
    \delta_{jk}\delta^{(3)}(\textbf{p}-\textbf{q}), 
\end{align}
and the same with $(a\leftrightarrow b)$; thus
\begin{align}
    \left[Q^{\mu},Q^{\nu}\right] &= \int \frac{\mathop{d^3p} }{(2\pi)^3}
    a_{\textbf{p}i}^\dagger (s^{\mu})^{ij} \delta_{jk}
    (s^{\nu})^{kl}a_{\textbf{p}l} - a_{\textbf{p} k}^\dagger
    (s^{\nu})^{kl}\delta_{il} (s^{\mu})^{ij}a_{\textbf{p}j} + (a\leftrightarrow
    b)\nonumber \\ 
    &= \int \frac{\mathop{d^3p}}{(2\pi)^3}a^\dagger _{\textbf{p}i}\tensor[]{(s^{\mu})}{^{i}_k}
    (s^{\nu})^{kl} a_{\textbf{p}l} - a^\dagger _{\textbf{p} k}
    \tensor[]{(s^{\nu})}{^{k}_i} (s^{\mu})^{ij}a_{\textbf{p}j} + (a\leftrightarrow b).
\end{align}
Relabelling dummy indices in the second term: $k \leftrightarrow i , l \leftrightarrow j$, 
\begin{align}
    \left[Q^{\mu},Q^{\nu}\right] &= \int \frac{\mathop{d^3p}}{(2\pi)^3}
\left(     a_{\textbf{p}i}^\dagger  \tensor[]{\left[s^{\mu},
    s^{\nu}\right]}{^{il}}a_{\textbf{p}l}  +   b_{\textbf{p}i}^\dagger  \tensor[]{\left[s^{\mu},
    s^{\nu}\right]}{^{il}}b_{\textbf{p}l} \right)  . \label{2.38}
\end{align}
Our generators satisfy well-known commutator relations:
\begin{align}
\left[s^{0},s^{i}\right] &=   \left[\mathbb{I}, \frac{1}{2}\sigma^{i}\right]= 0\nonumber \\ 
\left[s^{i},s^{j}\right]&= \frac{1}{4} \left[\sigma^{i},\sigma^{j}\right] =
i\tensor[]{\varepsilon}{^{ij}_k} \frac{1}{2}\sigma^{k} =
i\tensor[]{\varepsilon}{^{ij}_k}s^{k}.
\end{align}
When we insert these into \eqref{2.38}, we get
\begin{align}
    \boxed{
        \left[Q,Q^{i}\right] = 0 
        }
\end{align}
\begin{align}
 \boxed{
     \left[Q^{i},Q^{j}\right] = i\tensor[]{\varepsilon}{^{ij}_k} Q^{k}.
     }
\end{align}
The conserved charges $Q^{i}$ satisfy the same Lie bracket as the generators of $SU(2)$.
Thus they are generators of a representation of $SU(2)$.
\section{Exercise 3}%
\begin{exercise}
   Starting from a real vector field $A^\mu$, the Proca density of Lagrangian is obtained
by adding to the Maxwell Lagrangian a mass term:
\begin{align}
   \mathcal{L}  = -\frac{1}{4}F_{\mu\nu}F^{\mu\nu} + \frac{1}{2}m^2A_\mu A^{\mu}
\end{align}
with $m >0$. Discuss the quantization of such a theory. This exercise can
be solved in different ways and the student can choose one of them (e.g. the
canonical quantization, as done in the lectures for the electromagnetic field, or
the particle quantization, as done for the scalar field). It is important to discuss:
\begin{itemize}
\item 
{\color{NavyBlue}
the gauge invariance of the Lagrangian: is this Lagrangian invariant under
    $A_\mu \to  A_\mu + \partial _\mu f $?;
    \item the equations of motion and their solutions;
    \item the physical degrees of freedom of the quantized particles;
    \item the commutation (or anticommutation) relation in spacetime and the
    Feynman propagator. 
}
\end{itemize}
\end{exercise}
\subsection{Gauge invariance}%
I will proceed with a canonical quantisation approach. For starters, the given Lagrangian
is not gauge-invariant: upon 
\begin{align}
    A^{\mu}\to  A^{\mu}+ \partial ^{\mu} f,
\end{align}
the Lagrangian density becomes
\begin{align}
    \mathcal{L} &\to  -\frac{1}{4} \left(A_{\mu,\nu} + \cancel{f_{,\mu\nu}} - A_{\nu,\mu} -
\cancel{    f_{,\nu\mu}} \right) \left(A^{\mu,\nu} + \cancel{f^{,\mu\nu} }- A^{\nu,\mu} -
\cancel{f^{,\nu\mu}}\right)
    + \frac{1}{2} m^2 \left(A_\mu + f_{,\mu}\right) \left(A^{\mu}+ f^{,\mu}\right)\nonumber \\ 
    &= \mathcal{L} + \frac{1}{2}m^2 \left(A_\mu f^{,\mu} + f_{,\mu} A^{\mu}\right)\nonumber \\ 
    &= \mathcal{L} + m^2 A^{\mu}f_{,\mu} \neq \mathcal{L}.
\end{align}
This lack of gauge-invariance means that the gauge degree of freedom has
already been removed. All this can be justified, starting from the EOM. 

\subsection{Equations of motion}%
I will recycle
some of the calculations from the first problem. The Maxwell term
in the Lagrangian is the same as in \eqref{1.3}:
\begin{align}
    \mathcal{L} = -\frac{1}{2}A_{\mu,\nu}A^{\mu,\nu} + \frac{1}{2}A_{\mu,\nu}A^{\nu,\mu}+ \frac{1}{2}m^2 A_\mu A^{\mu}.
\end{align}
For the EL equations: the conjugate momenta are:
    \begin{align}
 \frac{\partial \mathcal{L}}{\partial \left(A_{\mu,\nu}\right)} &= -\frac{1}{2} \frac{\partial
 }{\partial\left(A_{\mu,\nu}\right)}\left(A_{\rho,\sigma}A^{\rho,\sigma} -
 A_{\sigma,\rho}A^{\rho,\sigma} - m^2 A_\mu A^{\mu}\right)\nonumber \\ 
&= -2 \frac{1}{2} \left(\delta_{\rho}^{\mu}\delta_{\sigma}^{\nu}A^{\rho,\sigma} -
\delta_{\sigma}^{\mu}\delta_{\rho}^{\nu}A^{\rho,\sigma}\right)\nonumber \\ 
&= -A^{\mu,\nu} + A^{\nu,\mu} . \label{3.46}
    \end{align}
For EL we differentiate again:
\begin{align}
    \implies \partial _\nu \frac{\partial \mathcal{L}}{\partial
    \left(A_{\mu,\nu}\right)}   &= -\tensor[]{A}{^{\mu}^{,\nu}_\nu}  +
    \tensor[]{A}{^{\nu,\mu}_\nu};   \label{3.47}
    \end{align}
the other term in EL is
    \begin{align}
    \frac{\partial \mathcal{L}}{\partial A_\mu} &= \frac{1}{2}m^2 2 A^{\mu} = m^2
    A^{\mu}.
\end{align}
The equations of motion then read
\begin{align}
\square A^{\mu} - \tensor[]{A}{^{\nu,\mu}_\nu} + m^2 A^{\mu}&=0.
\end{align}
As suggested, taking the four-divergence results in a further constraint:
\begin{align}
\cancel{   \tensor[]{A}{^{\mu,\nu}_\mu}}  - \cancel{\tensor[]{A}{^{\nu,\mu}_\nu}} + m^2
   \tensor[]{A}{^{\mu}_{,\mu}} = 0.
\end{align}
This is just the Lorenz gauge condition:
\begin{align}
   \boxed{
       \tensor[]{A}{^{\mu}_{,\mu}}  = 0,
       }
\end{align}
which explains why gauge invariance is broken; in this gauge, the equations of motion reduce to Klein-Gordon:
\begin{align}
\boxed{
    \square A^{\mu}  + m^2 A^{\mu}=0.
    }
\end{align}
Moreover: from group theory we know that the 4-vector representation of the Lorentz group
is $(1 /2, 1 /2) = \textbf{0}\oplus \textbf{1}$. Now, the Lorenz gauge, being a Lorentz-invariant
condition, is also rotation invariant. But the
degrees of freedom of an $SO(3)$ irreducible representation mix under rotations. Thus the
constraint has to remove all of them, i.e. a complete representation. But it is only one
constraint, thus it can only remove one degree of freedom. The only possibility is that it
removes the scalar representation! In other words, whatever we quantise we are sure will
be
a massive spin-1 field.

The solution to Klein-Gordon can be found as usual by moving into momentum space:
\begin{align}
    \left(\square + m^2\right)A^{\mu}&= 0 \nonumber \\ 
   \int \frac{\mathop{d^4p}}{(2\pi)^4} (-p^2 + m^2) \tilde{A}^{\mu}e^{-i p\cdot x} &= 0.
\end{align}
This is the on-shell condition, satisfied for $\tilde{A}^{\mu}(p) = \delta(-p^2+m^2)C^{\mu}(p) $. 
\begin{align}
    A^{\mu}=  \int \frac{\mathop{d^4p}}{(2\pi)^4} C^{\mu}(p) \delta (-p^2 + m^2) e^{-i
    p\cdot x}
\end{align}
Then we apply the celebrated delta function identity 
\begin{align}
    \delta(f(x)) = \sum_{i = \text{zeros}}
    \frac{\delta(x-x_{i})}{\left|f^\prime(x_{i})\right|}.
\end{align}
In our case, $f(p^{0}) = - (p^{0})^2 + \textbf{p}^2+ m^2 $, $f^\prime = -2p^{0}$, and the two
zeros are $p^{0}=  \pm (\textbf{p}^2 + m^2)^{1 /2}$. Thus
\begin{align}
    A^{\mu}=  \int \frac{\mathop{d^4p}}{(2\pi)^4} C^{\mu}(p) \frac{1}{2E_p}
    \left[\delta\left(p^{0} - E_p \right) + \delta \left(p^{0}+
    E_p\right)\right]e^{-ip\cdot x}.
\end{align}
Changing variable in the second term: $p^{0}\to -p^{0}$; performing the integral in $\mathop{dp^{0}} $
\begin{align}
    A^{\mu} = \int \frac{\mathop{d^3p}}{(2\pi)^3} \frac{1}{2E_p}
    \left( \frac{C^{\mu}(E_p,\textbf{p})}{2\pi} e^{-ip\cdot x}  +
    \frac{C^{\mu}(-E_p,\textbf{p})}{2\pi}e^{ip\cdot x}\right) _{p^{0}= E_p}.
\end{align}
For the field to be real it must hold that $C^{\mu} (-E_p ) = (C^{\mu}(E_p))^\star$. As
usually done in the lecture, the quantum theory is more conveniently developed by choosing
$C^{\mu}$ such that
\begin{align}
    A^{\mu} = \int \frac{\mathop{d^3p}}{(2\pi)^3} \frac{1}{\sqrt{2E_p} }
\sum_\lambda    \left( a_{\lambda,\textbf{p}} \varepsilon^{\mu}_{\lambda} (p)e^{-ip\cdot x}  +
    a_{\lambda,\textbf{p}}^\star (\varepsilon_\lambda^{\mu}(p))^\star e^{ip\cdot x}\right) _{p^{0}=
    E_p}. \label{3.59}
\end{align}
What I did was simply reabsorb some coefficients and decompose $C^{\mu}$ on a basis of
polarisation four-vectors
$\varepsilon_\lambda (p)$. The Lorenz gauge condition tells us that
\begin{align}
0= \tensor[]{A}{^{\mu}_{,\mu}} = \int \frac{\mathop{d^3p}}{(2\pi)^3}\frac{1}{\sqrt{2E_p}
}\sum_\lambda \left(a_\lambda \varepsilon^{\mu}_\lambda(p) (-i p_\mu) e^{-ip\cdot x} + \text{c.c.}\right)
\end{align}
or 
\begin{align}
    p_\mu \varepsilon^{\mu}_\lambda (p) = 0.
\end{align}
As foreshadowed, this takes one basis vector out of the game. I leave here an identity
which I shall employ later.
\begin{rmk}[Sum rule]
If I pick a normalisation $(\varepsilon_\lambda)^{\mu} (\varepsilon_\lambda^\star)_{\mu} =
-1$, then
\begin{align}
\boxed{
    \sum _\lambda \left(\varepsilon_\lambda\right)_\mu (p)
    \left(\varepsilon_\lambda^\star\right)_\nu(p) = -g_{\mu\nu} + \frac{p_\mu
    p_\nu}{m^2}\label{3.62}
    }
\end{align}
\end{rmk}
\begin{proof}
    I will compute the LHS in a given frame and then write the result in a covariant
    form. If $p^{\mu} = (E_p,0,0,p_z)$, then I can pick
    \begin{align}
    \varepsilon_1 &= (0,1,0,0)\nonumber \\ 
    \varepsilon_2 &= (0,0,1,0)\nonumber \\ 
    \varepsilon_3 &= \frac{1}{m}(p_z,0,0,E_p).
    \end{align}
    These vectors satisfy my normalisation (there appears a minus sign on the spatial
    components when taking the Minkowski product, plus $p_z^2 - E_p^2= -m^2$). Then
    \begin{align}
        \sum_\lambda (\varepsilon_\lambda)_\mu (\varepsilon_\lambda ^\star)_\nu
        &=
        \begin{pmatrix} 0\\ 1 \\0 \\0 \end{pmatrix} \begin{pmatrix} 0 & 1 & 0 & 0
        \end{pmatrix} + 
        \begin{pmatrix} 0\\ 0 \\1 \\0 \end{pmatrix} \begin{pmatrix} 0 & 0 & 1 & 0
        \end{pmatrix} + 
  \frac{1}{m^2}      \begin{pmatrix} p_z\\ 0 \\0 \\E_p \end{pmatrix} \begin{pmatrix} p_z & 0
  & 0 & E_p
        \end{pmatrix} \nonumber \\ 
        &= \begin{pmatrix} 0&&&\\&1&&\\&&0&\\&&&0 \end{pmatrix} +\begin{pmatrix}
        0&&&\\&0&&\\&&1&\\&&&0 \end{pmatrix} + \frac{1}{m^2}\begin{pmatrix} p_z^2
        &&&p_zE_p\\&0&&\\&&0&\\p_zE_p&&&E_p^2 \end{pmatrix} .
    \end{align}
    Now, on the last term: $E_p^2 - p_z ^2 = m^2$ or
    \begin{align}
        \frac{p_z^2}{m^2} &= \frac{E_p^2}{m^2}-1,\nonumber \\ 
        \frac{E_p^2}{m^2} &= \frac{p_z^2}{m^2}+1 .   
        \end{align}
        Thus
        \begin{align}
        \sum_\lambda (\varepsilon_\lambda)_\mu (\varepsilon_\lambda ^\star)_\nu
        &= \begin{pmatrix} -1 &&&\\&1&&\\&&1&\\&&&1 \end{pmatrix} +
        \frac{1}{m^2}\begin{pmatrix} E_p^2 &&&p_zE_p \\&0&&\\&&0&\\p_zE_p &&&p_z^2
        \end{pmatrix} \nonumber \\ 
        &= -g_{\mu\nu} + \frac{1}{m^2}p _\mu p_\nu.
        \end{align}
        The equation is now Lorentz covariant, thus it must hold in any frame.
\end{proof}
\subsection{Canonical quantisation}%
For a healthy canonical quantisation, we need to check whether the Hamiltonian formalism
of this theory is sound. In particular, I wish to prove that
\begin{enumerate}
    \item the Hamiltonian density is positive-definite (this should safeguard us against the existence
    of arbitrarily-negative enrgy states after quantisation);
    \item the Hamiltonian formalism is equivalent to the Lagrangian one.
\end{enumerate}
The conjugate momenta I find from \eqref{3.47}:
\begin{align}
    \Pi^{\mu} = \frac{\partial \mathcal{L}}{\partial \left(A_{\mu,0}\right)} =
    -A^{\mu,0}+ A^{0,\mu} = -F^{0\mu}.
\end{align}
Thus arises an issue, in that $\Pi^{0} = 0$. Ideally, $A^{0}$ decouples from the dynamics,
for then we need not quantise it. Let us try to write a Hamiltonian (density):
\begin{align}
    \mathcal{H} &= \Pi_0 \dot{A}^{0} + \Pi_{i}\dot{A}^{i} - \mathcal{L}\nonumber \\ 
    &= \Pi_0 \dot{A}^{0} + \Pi_i \dot{A}^{i} + \frac{1}{4}F_{\mu\nu}F^{\mu\nu} -
    \frac{1}{2}m^2 A_\mu A^{\mu}.
\end{align}
Expanding the Maxwell term:
\begin{align}
\frac{1}{4}F_{\mu\nu}F^{\mu\nu} &= \frac{1}{4} \left(F_{0\nu}F^{0\nu} +
F_{i\nu}F^{i\nu}\right) \nonumber \\ 
&= \frac{1}{4}(\underbrace{F_{0i}F^{0i}}_{(-)^2 \Pi_i \Pi^{i}} + \underbrace{F_{i0}F^{i
0}}_{\Pi_i \Pi^{i}} + F_{ij}F^{ij}),
\end{align}
and 
\begin{align}
    F_{ij}F^{ij}&=  \left(A_{i,j}- A_{j,i}\right)\left(A^{i,j}-A^{j,i}\right)\nonumber \\ 
&= A_{i,j}A^{i,j}+A_{j,i}A^{j,i} - A_{i,j}A^{j,i}- A_{j,i}A^{i,j}\nonumber \\ &= 2
\left(A_{i,j}A^{i,j}- A_{i,j}A^{j,i}\right). \label{3.70}
\end{align}
Actually, for the moment I will need this term in a different form.  $F_{ij}=  A_{[i,j]}=
\varepsilon_{ijk} (\boldsymbol{\nabla}\times \textbf{A})^{k}$ and thus
\begin{align}
    F_{ij}F^{ij} &= \varepsilon_{ijk}\tensor[]{\varepsilon}{^{ij}_l}
    \left(\boldsymbol{\nabla}\times \textbf{A}\right)^{k}\left(\boldsymbol{\nabla}\times
    \textbf{A}\right)^{l}\nonumber \\ 
    &= 2\delta_{kl}\left(\boldsymbol{\nabla}\times \textbf{A}\right)^{k}\left(\boldsymbol{\nabla}\times
    \textbf{A}\right)^{l}\nonumber \\ 
    &= 2 \left(\boldsymbol{\nabla}\times \textbf{A} \right)^2,\label{3.71}
\end{align}
where I used the identity $\varepsilon_{ijk}\tensor[]{\varepsilon}{^{ij}_l} =
2\delta_{kl}$.

Recollecting all the bits,
\begin{align}
    \mathcal{H} = \Pi_0 \dot{A}^{0} + \Pi_i \dot{A}^{i} + \frac{1}{2}\Pi_i \Pi^{i} +
    \frac{1}{2}\left(\boldsymbol{\nabla}\times \textbf{A}\right)^2 - \frac{1}{2}m^2 \left(
    A^{0} \right) ^2 + \frac{1}{2}m^2 \left( A^{i} \right) ^2.
\end{align}
I have to get rid of $\dot{A}^{\mu}$. For $\dot{A}^{i}, $ I can invert the relation for
$\Pi^{i}= -\dot{A}^{i}+A^{0,i}$:
\begin{align}
    \mathcal{H} &= \Pi_0 \dot{A}^{0} + \Pi_i (-\Pi^{i}+ \tensor[]{A}{^{0}^{,i}}) + \frac{1}{2}\Pi_i \Pi^{i} +
    \frac{1}{2}\left(\boldsymbol{\nabla}\times \textbf{A}\right)^2 - \frac{1}{2}m^2
    \left(
    A^{0} \right) ^2 + \frac{1}{2}m^2 \left( A^{i} \right) ^2\nonumber \\ 
    &= \Pi_0 \dot{A}^{0}(\Pi_{k},A^{k}) - \frac{1}{2}\Pi_i \Pi^{i} + \Pi_i \tensor[]{A}{^{0}^{,i}} +
    \frac{1}{2}\left(\boldsymbol{\nabla}\times \textbf{A}\right)^2 - \frac{1}{2}m^2
    \left(
    A^{0} \right) ^2 + \frac{1}{2}m^2 \left( A^{i} \right) ^2.
\end{align}
So far, the Hamiltonian formalism is problematic. Hamilton's equations are
\begin{align}
&    \dot{A}^{i} = \frac{\delta \mathcal{H}}{\delta \Pi_i} , & & \dot{A}^{0} =
\frac{\delta \mathcal{H}}{\delta \Pi_0} = 0, \nonumber \\ 
&    \dot{\Pi}_{i} = -\frac{\delta \mathcal{H}}{\delta A^i} , & & \dot{\Pi}_{0} =
-\frac{\delta \mathcal{H}}{\delta A^{0}} \neq  0.
\end{align}
Although $\Pi_0$ is identically zero, it evolves! The source of this discrepancy is the
fact that I have kept $A^{0}$ as a dynamical variable, whereas its dynamics is determined
by the other components (we do have 3 degrees of freedom, after all). The EL equation for
$A^{0}$ reads
\begin{align}
0 &=\partial _\nu \frac{\partial \mathcal{L}}{\partial A_{0,\nu}} - \frac{\partial
\mathcal{L}}{\partial A_{0}} ;
\end{align}
from \eqref{3.46} I know that $\partial \mathcal{L} / \partial  (A_{0,\nu}) = -A^{0,\nu} +
A^{\nu,0} = F^{0,\nu} = - \Pi^{\nu}$:
\begin{align}
0&= -\cancel{\partial _0 \Pi^{0}}- \partial _i \Pi^{i} - m^2A^{0}  
\end{align}
or 
\begin{align}
    A^{0}= -\tensor[]{\Pi}{^{i}_{,i}} / m^2. \label{3.77}
\end{align}
Now I can remove $\Pi_0$ and substitute
$A^{0}$ out from the Hamiltonian:
\begin{align}
    \mathcal{H}&= \cancel{\Pi_0 \dot{A}^{0}(\Pi_{k},A^{k})} - \frac{1}{2}\Pi_i \Pi^{i} + \Pi_i \tensor[]{A}{^{0}^{,i}} +
    \frac{1}{2}\left(\boldsymbol{\nabla}\times \textbf{A}\right)^2 - \frac{1}{2}m^2
    \left(
    A^{0} \right) ^2 + \frac{1}{2}m^2 \left( A^{i} \right) ^2\nonumber \\ 
    &= -\frac{1}{2}\Pi_i \Pi^{i} - \Pi_i
   \frac{1}{m^2} (\tensor[]{\Pi}{^{j}^{,i}_j})+\frac{1}{2}\left(\boldsymbol{\nabla}\times
    \textbf{A}\right)^2 - \frac{1}{2}\frac{m^2}{m^4} \left(\tensor[]{\Pi}{^{i}_{,i}}\right)^2 +
    \frac{1}{2}m^2 (A^{i})^2.
\end{align}
I lower all the conjugate momenta spatial indices, paying a minus sign each time:
\begin{align}
    \mathcal{H}
    &= +\frac{1}{2}\Pi_i \Pi_{i} - \Pi_i
   \frac{1}{m^2} (\tensor[]{\Pi}{_{j}_{,j}_i})+\frac{1}{2}\left(\boldsymbol{\nabla}\times
    \textbf{A}\right)^2 - \frac{1}{2m^2} \left(\tensor[]{\Pi}{_{i}_{,i}}\right)^2 +
    \frac{1}{2}m^2 (A^{i})^2.
\end{align}
Integrating by parts the second term ($\Pi_i \Pi_{j,ij} = - \Pi_{i,i} \Pi_{j,j}$) and
simplifying it with the fourth:
\begin{align}
    \boxed{
        \mathcal{H} = \frac{1}{2}\left( \Pi_i  \right)^2 + \frac{1}{2m^2}\left( \Pi_{i,i} \right) ^2
        +\frac{1}{2} \left(\boldsymbol{\nabla}\times \textbf{A}\right)^2 + \frac{1}{2}m^2
        \left(A^{i}\right)^2.
        }
\end{align}
\begin{rmk}
    This Hamiltonian density is positive-definite! 
\end{rmk}
With that out of the way, I now show that
the Hamiltonian formalism gives back the correct equations of motion. 
\begin{rmk}
   Hamilton's equations imply EL. 
\end{rmk}
\begin{proof}
Changing back the
third term $(\boldsymbol{\nabla}\times \textbf{A})^2 = (A_{i,j}A^{i,j} - A_{i,j}A^{j,i})$
(from equations \eqref{3.70}, \eqref{3.71})
\begin{align}
        \mathcal{H} = \frac{1}{2}\left( \Pi_i  \right)^2 + \frac{1}{2m^2}\left( \Pi_{i,i} \right) ^2
       +\frac{1}{2} (A_{i,j}A^{i,j}- A_{i,j}A^{j,i})+ \frac{1}{2}m^2
        \left(A^{i}\right)^2,
\end{align}
Hamilton's equations (in functional form) read
\begin{align}
    \dot{\Pi}_i &= - \frac{\delta \mathcal{H}}{\delta A^{i}}\nonumber \\ 
    &= -\left( \frac{\partial }{\partial A^{i}}  - \partial _j \frac{\partial }{\partial
    (\partial _j A^{i}) }  \right)  \mathcal{H} \nonumber \\ 
    &= - m^2 A_{i} + \frac{1}{2}\partial _j \frac{\partial
    }{\partial(\tensor[]{A}{^{i}_{,j}})  }  \left(\tensor[]{A}{_{k,}^{l}}
    \tensor[]{A}{^{k}_{,l}} - \tensor[]{A}{^{k}_{,l}}\tensor[]{A}{^{l}_{,k}}\right) \nonumber \\ 
    &= -m^2 A_{i} + \frac{1}{2}\partial _j 2\left( \tensor[]{A}{_{i,}^{j}} -
    \tensor[]{A}{^{j}_{,i}}\right) \nonumber \\ 
    &= -m^2 A_{i} + \tensor[]{A}{_{i,}^{j}_j} - \tensor[]{A}{^{j}_{,ji}};\label{3.81}
\end{align}
and
\begin{align}
    \dot{A^{i}}&= \frac{\delta \mathcal{H}}{\delta \Pi_i} \nonumber \\ 
    &= \left(\frac{\partial }{\partial \Pi_i} - \partial _j \frac{\partial }{\partial
    (\partial _j \Pi_i) }  \right) \mathcal{H}\nonumber \\ 
    &= \Pi_i - \frac{1}{2m^2} \partial _j \frac{\partial }{\partial \Pi_{i,j}}
    \left(\Pi_{l,l}\Pi_{k,k}\right)\nonumber \\ 
    &= \Pi_i - \frac{1}{m^2}\partial _j \Pi_{k,k}\delta_{ij}\nonumber \\ 
    &= \Pi_i - \frac{1}{m^2}\Pi_{k,ki}. \label{3.82}
\end{align}
One sees that this set of equations implies EL by substituting \eqref{3.77}: $A^{0} =
\tensor[]{\Pi}{^{i}_{,i}} / m^2 = + \Pi_{i,i} / m^2$
\begin{align}
\tensor[]{A}{^{i}_{,0}}&= \Pi_{i} - \tensor[]{A}{^{0}_{,i}},\nonumber \\ 
\tensor[]{A}{^{i}_{,00}}&= \Pi_{i,0} - \tensor[]{A}{^{0}_{,0i}},
\end{align}
and using \eqref{3.81} for $\Pi_{i,0}$:
\begin{align}
    \tensor[]{A}{^{i}_{,00}} &= -m^2 A_i + \tensor[]{A}{_{i,}^{j}_j} - \tensor[]{A}{^{j}_{,ji}}
    -\tensor[]{A}{^{0}_{,0i}}\nonumber \\ 
    -\tensor[]{A}{_{i,}^{0}_0} &= -m^2 A_i + \tensor[]{A}{_{i,}^{j}_j} - \tensor[]{A}{^{j}_{,ji}}
    -\tensor[]{A}{^{0}_{,0i}}.
\end{align}
Rearranging
\begin{align}
\tensor[]{A}{_{i,}^{0}_0} + \tensor[]{A}{_{i,}^{j}_j} - \tensor[]{A}{^{0}_{,0i}} -
\tensor[]{A}{^{j}_{,ji}}  - m^2 A_i&=0\nonumber \\ 
\square A_i - \tensor[]{A}{^{\mu}_{,\mu i }} - m^2 A_i &= 0
\end{align}
which is EL for the spatial components of $A_\mu$ (the dynamical components). 
\end{proof}
\subsubsection{Commutator relations}%
We are working with a bosonic field, which we know translates to commutator relations. Promoting $A^{i},\Pi_j$ to operators, I
would like them to satisfy the equal-time commutator relations:
\begin{align}
    \left[\hat{A}^{i}(x), \hat{\Pi}_j (y)\right]_{x^{0}= y^{0}} &\overset{!}{=} i \delta^{i}_j
    \delta^{(3)}(\textbf{x}-\textbf{y})\nonumber \\ 
    \left[\hat{A}^{i}(x), \hat{A}^j (y)\right]_{x^{0}= y^{0}} &\overset{!}{=} 0 \nonumber \\ 
    \left[\hat{\Pi}_{i}(x), \hat{\Pi}_j (y)\right]_{x^{0}= y^{0}} &\overset{!}{=} 0.\label{3.87}
\end{align}
From an axiomatic standpoint I suppose it be best that I postulate these commutator
relations and derive the ladder operator algebra as a consequence. I shall do the inverse,
for the computation is significantly smoother (analogously to what was done in class). In
particular: the expression for the field \eqref{3.59} is modified, promoting $A^{\mu}$ and
thus $a_\lambda$ to operators:
\begin{align}
    \hat{A}^{\mu} = \int \frac{\mathop{d^3p}}{(2\pi)^3} \frac{1}{\sqrt{2E_p} }
\sum_\lambda    \left( \hat{a}_{\lambda\textbf{p}}  \varepsilon^{\mu}_{\lambda} (p)e^{-ip\cdot x}  +
    \hat{a}_{\lambda \textbf{p}}^\dagger  (\varepsilon_\lambda^{\mu}(p))^\star e^{ip\cdot x}\right)
    _{p^{0}= E_p};
\end{align}
(dropping the hats from now on); I show now that
\begin{prop}
    If the operators $a_{\lambda \textbf{p}}, a_{\lambda \textbf{p}}^\dagger$ satisfy the following
    algebra:
    \begin{align}
     [a_{\lambda\textbf{p}}, a_{\lambda^\prime \textbf{q}}^\dagger ]   &=
    (2\pi)^3 \delta^{(3)}(\textbf{p}-\textbf{q}) \delta_{\lambda\lambda^\prime},\nonumber \\ 
     \left[a_{\lambda\textbf{p}}, a_{\lambda^\prime \textbf{q}}\right]   &=0
    \end{align}
    then \eqref{3.87} is satisfied.
\end{prop}
\begin{proof}
   Starting off easy:
   \begin{align}
       [A^{i},A^{j}] = \int \frac{\mathop{d^3p}}{(2\pi)^3}\int
       \frac{\mathop{d^3q}}{(2\pi)^3}\frac{1}{\sqrt{2E_p 2E_q}
       }\sum_{\lambda,\lambda^\prime} \left[a_{\lambda
       \textbf{p}}\varepsilon_{\lambda}^{i} e^{-ip\cdot x} +\text{h.c.}, a_{\lambda^\prime
       \textbf{q}}\varepsilon_{\lambda^\prime}^{j} e^{-iq\cdot y} +\text{h.c.}\right].
   \end{align}
   The commutators of the type $[a,a] $ and  $[a^\dagger ,a^\dagger ] $ vanish. All is left is
   \begin{align}
       [A^{i},A^{j}] = \int \frac{\mathop{d^3p}}{(2\pi)^3}\int
       \frac{\mathop{d^3q}}{(2\pi)^3}\frac{1}{\sqrt{2E_p2E_q}
       }\sum_{\lambda,\lambda^\prime}&\left[a_{\lambda \textbf{p}}, a_{\lambda^\prime
       \textbf{q}}^\dagger\right]\varepsilon_\lambda^{i}(p)\left( \varepsilon_{\lambda^\prime}
       ^\star\right) ^{j}(q)e^{-ip\cdot x}e^{iq\cdot y} +\nonumber \\ +& \left[a_{\lambda \textbf{p}}^\dagger , a_{\lambda^\prime
       \textbf{q}}\right]\left( \varepsilon_\lambda ^\star\right) ^{i}(p)
       \varepsilon_{\lambda^\prime}
        ^{j}(q)e^{ip\cdot x}e^{-iq\cdot y} .
   \end{align}
   These cancel exactly: the LHS is anti-hermitian ($[A^{i},A^{j}]^\dagger  =
   \left(A^{i}A^{j}- A^{j}A^{i}\right)^\dagger  = ( A^{j}) ^\dagger (A^{i})^\dagger -
   (A^{i})^\dagger (A^{j})^\dagger  = -[A^{i,}A^{j}]$) and the RHS is hermitian: the
   commutator contains a $\delta_{\lambda\lambda^\prime}$ and a
   $\delta^{(3)}(\textbf{p}-\textbf{q})$, reducing the polarisation sum
   to the one in \eqref{3.62}, which is symmetric under $i\leftrightarrow j$. Thus the second term is the
   h.c. of the first. The only operator that is both hermitian and anti-hermitian is 0.
   %the commutators force $\lambda = \lambda^\prime$ and $\textbf{p}
   %= \textbf{q}$. Then we get a polarisation sum of the type \eqref{3.62}, symmetric upon
   %$i\leftrightarrow j$. The remaining momentum space integral takes care of the plane
   %waves, turning them into the same Dirac delta. The cancellation is due to the second
   %commutator picking up a minus sign due to antisymmetry.

   For the other field commutators, we need to evaluate $\Pi_i$
   \begin{align}
       \Pi_i &= -\dot{A}_{i} + A_{0,i} \nonumber \\ 
       &= \int \frac{\mathop{d^3p}}{(2\pi)^3} \frac{1}{\sqrt{2E_p} } \sum_\lambda
       \left[a_{\lambda \textbf{p}}\left(-\varepsilon_{\lambda i}\partial _t +
       \varepsilon_{\lambda 0}\partial _i  \right)e^{-ip\cdot x}+ \text{h.c.}\right]\nonumber \\ 
       &= \int \frac{\mathop{d^3p}}{(2\pi)^3}\frac{1}{\sqrt{2E_p} }\sum_\lambda
       [a_{\lambda \textbf{p}}(-\varepsilon_{\lambda i} (-iE_p ) + \varepsilon_{\lambda 0}
       ip_i) e^{-ip\cdot x}+\text{h.c.}]\nonumber \\ 
       &\coloneqq i\int \frac{\mathop{d^3p}}{(2\pi)^3}\sqrt{\frac{E_p}{2}} \sum_\lambda
       [a_{\lambda \textbf{p}}\tilde{\varepsilon}_{\lambda i} e^{-ip\cdot x}-\text{h.c.}]
   \end{align}
   where $+\text{h.c.}$ becomes $-\text{h.c.}$ after grouping $i$ outside, and I have defined 
   \begin{align}
      \tilde{ \varepsilon}_{\lambda i}(p) \coloneqq \varepsilon_{\lambda i}(p) +
      \frac{p_i}{E_p}\varepsilon_{\lambda 0 }(p).
   \end{align}
   The commutator  $[\Pi_i,\Pi_j]$ is worked out in the same way as the previous one: the
   terms with $[a,a]$ and  $[a^\dagger ,a^\dagger ] $ die and all is left is
   \begin{align}
       [\Pi_i, \Pi_j] = -i^2\int \frac{\mathop{d^3p}}{(2\pi)^3}\int
       \frac{\mathop{d^3q}}{(2\pi)^3} \sqrt{\frac{E_p E_q}{4}} 
       \sum_{\lambda,\lambda^\prime}&\left[a_{\lambda \textbf{p}}, a_{\lambda^\prime
       \textbf{q}}^\dagger\right]\left( \tilde{\varepsilon}_\lambda \right) _{i}\left( \tilde{\varepsilon}_{\lambda^\prime}
       ^\star\right) _{j}e^{-ip\cdot x}e^{iq\cdot y} +\nonumber \\ +& \left[a_{\lambda \textbf{p}}^\dagger , a_{\lambda^\prime
       \textbf{q}}\right]\left( \tilde{\varepsilon}_\lambda ^\star\right) _{i}
       \left( \tilde{\varepsilon}_{\lambda^\prime} \right) 
        _{j}e^{ip\cdot x}e^{-iq\cdot y} .
   \end{align}
   If the new polarisation sum is symmetric under $i\leftrightarrow j$, we can make the
   same argument as before. The new polarisation sum is
   \begin{align}
   \sum _\lambda (\tilde{\varepsilon}_\lambda)^{i} (\tilde{\varepsilon}_\lambda^\star)^j
   &=\sum_\lambda \left(\varepsilon_\lambda ^{i} +\frac{p^i}{E_p}\varepsilon_\lambda^{0}\right)
   \left(\left( \varepsilon_\lambda ^\star\right)^{j} +
   \frac{p^j}{E_p}\left(\varepsilon_\lambda^\star\right)^{0} \right)\nonumber \\ 
   &= \sum_\lambda\varepsilon_{\lambda}^{i}\left(\varepsilon_\lambda^\star\right)^{j} +
   \frac{p^i}{E_p}\varepsilon_{\lambda}^{0}\left(\varepsilon_\lambda^\star\right)^{j} +
   \frac{p^j}{E_p}\varepsilon_\lambda^{i} \left(\varepsilon_\lambda^\star\right)^{0} +
   \frac{p^{i}p^{j}}{E_p^2}
   \varepsilon_\lambda^{0}\left(\varepsilon_\lambda^\star\right)^{0}\nonumber \\ 
   &= \sum_\lambda\varepsilon_{\lambda}^{i}\left(\varepsilon_\lambda^\star\right)^{j} +
   \frac{p^i}{E_p}\varepsilon_{\lambda}^{0}\left(\varepsilon_\lambda^\star\right)^{j} +
   \frac{p^j}{E_p}\varepsilon_\lambda^{0} \left(\varepsilon_\lambda^\star\right)^{i} +
   \frac{p^{i}p^{j}}{E_p^2}
   \varepsilon_\lambda^{0}\left(\varepsilon_\lambda^\star\right)^{0}.
   \end{align}
   The first and last term are symmetric, and so is the sum of the second with the third.
   Thus this commutator also vanishes.

   The last commutator is
   \begin{align}
       [A^{i},\Pi_j] = \int \frac{\mathop{d^3p}}{(2\pi)^3}\int
       \frac{\mathop{d^3p}}{(2\pi)^3}\frac{1}{\sqrt{2E_p} } i\sqrt{\frac{E_q}{2}}
       \sum_{\lambda,\lambda^\prime}\left[a_{\lambda\textbf{p}}\varepsilon_\lambda^{i}e^{-ip\cdot
       x}+\text{h.c.}, a_{\lambda^\prime
       \textbf{q}}(\tilde{\varepsilon}_{\lambda})_j e^{-iq\cdot y}-\text{h.c.}\right].
   \end{align}
   Again the $[a,a], [a^\dagger ,a^\dagger ]$ terms die:
   \begin{align}
       [A^{i},\Pi_j] = i\int \frac{\mathop{d^3p}}{(2\pi)^3}\int
       \frac{\mathop{d^3q}}{(2\pi)^3} \sqrt{\frac{E_q}{4E_p}}
       \sum_{\lambda,\lambda^\prime}\Big\{-&\left[a_{\lambda\textbf{p}},
       a_{\lambda^\prime\textbf{q}}^\dagger
       \right]\varepsilon_{\lambda}^{i}(p)\left(\tilde{\varepsilon}^\star_{\lambda^\prime}\right)_{j}(q)e^{-ip\cdot
       x} e^{iq\cdot y}
       + \nonumber \\ 
       +& \left[a_{\lambda\textbf{p}}^\dagger ,
       a_{\lambda^\prime\textbf{q}}
       \right]\left( \varepsilon_{\lambda} ^\star\right)
       ^{i}(p)\left(\tilde{\varepsilon}_{\lambda^\prime}\right)_{j}(q) e^{ip\cdot
       x}e^{-iq\cdot y}\Big\}.
   \end{align}
   I substitute the commutator relations:
   \begin{align}
       [A^{i},\Pi_j] = -i \int
       \frac{\mathop{d^3p}}{(2\pi)^3}\sqrt{\frac{\cancel{E_p}}{4\cancel{E_p}}}
       \sum_\lambda \left( \varepsilon_{\lambda}^{i} \left(\tilde{\varepsilon}_{\lambda}^\star\right)_j
       e^{-ip\cdot (x-y)} +
       \left(\varepsilon_{\lambda}^\star\right)^{i}\left(\tilde{\varepsilon}_\lambda\right)_j
       e^{ip\cdot (x-y)} \right) .
   \end{align}
   This calculation I have to do: 
   \begin{align}
       \sum_\lambda \varepsilon^{i}_{\lambda}\left(\tilde{\varepsilon}^\star_\lambda\right)_j &= 
       \sum_\lambda \varepsilon_\lambda^{i}\left[ \left(\varepsilon^\star_\lambda\right)_j  +
       \frac{p_j}{E_p}\left( \varepsilon _\lambda
^\star \right) _0 \right]     \nonumber \\ 
&= - g^{i}_j - \frac{p^{i}p_j}{m^2} - \frac{p_i}{E_p} \left(\cancel{g^{i}_0 }-
\frac{p^{j}p_0}{m^2}\right);
\end{align}
now, $p_0 = E_p$ (since we are on-shell) and the last two terms cancel. Moreover,
$g^{i}_j= \delta^{i}_j$ by definition. So the sum reduces to $-\delta^{i}_j$. The other polarisation sum gives the same value (it
is the complex conjugate of what I just found, which is a real number), cancelling the $1
/\sqrt{4}  = 1 /2$. The remaining
plane wave $\mathop{d^3p} $ integral yields a delta: 
\begin{align}
    [A^{i}, \Pi_j] = -i (-\delta^{i}_j) \delta^{(3)}(\textbf{x}-\textbf{y}).
\end{align}
\end{proof}
\subsection{Feynman propagator}%
At last I compute the Feynman propagator. In principle, that would require computing the
time-ordered product of fields. However, as a consequence of the Dyson-Schwinger
equation,\footnote{which in principle could be proven without the path integral
formulation, albeit I failed in doing so.}
in any theory the 2-point correlation function of fields can be expressed as the Green function of the EL
equations. If a current is introduced in the Lagrangian ($\mathcal{L}_\text{int} = -J_\mu
A^{\mu} $), the EL system becomes
\begin{align}
J_{\mu}&=    (\partial _\rho \partial ^{\rho} + m^2) A_{\mu} - \partial _{\mu}\partial _\nu
    A^{\nu}  \nonumber \\ 
    &= \left[\left(\square +m^2\right)g_{\mu\nu} - \partial _\mu \partial _\nu
    \right]A^{\nu}.
\end{align}
the Green's function is defined so that $A^{\mu}\overset{!}{=}-i G^{\mu\nu}\star J_\nu$
(where the $-i$ is put in place to be consistent with Dyson-Schwinger), or
\begin{align}
    i\delta_{\mu}^{\lambda}\delta^{(4)}(x-y)&\overset{!}{=}  \left[\left(\square +m^2\right)g_{\mu\nu} - \partial _\mu \partial _\nu
    \right]G^{\nu\lambda}.
\end{align}
I move to momentum space:
\begin{align}
    i\delta_{\mu}^{\lambda} = \left[\left(-p^2 + m^2\right)g_{\mu\nu} + p_\mu p_\nu\right]G^{\nu\lambda}.
\end{align}
Now, to invert this relation, one could first contract with $p^{\mu}$:
\begin{align}
ip^{\lambda} &= \left[\left(-p^2+m^2\right)p_{\nu} + p^2 p_\nu\right]G^{\nu\lambda}\nonumber \\ 
&= m^2 p_\nu G^{\nu\lambda},
\end{align}
thus I can substitute $ip^{\lambda} / m^2$ in place of $p_\nu G^{\nu\lambda}$:
\begin{align}
    i\delta_{\mu}^{\lambda} &= \left(-p^2 +m^2\right)g_{\mu\nu}G^{\nu\lambda} + i
    \frac{p_\mu p^{\lambda}}{m^2},
\end{align}
rearranging
\begin{align}
   g_{\mu\nu}G^{\nu\lambda} = i \frac{\delta_{\mu}^{\lambda}- p_\mu p^{\lambda} / m^2
   }{-p^2 + m^2}
\end{align}
and contracting with the metric
\begin{align}
    G^{\mu\lambda} = i \frac{g^{\mu\lambda} - p^{\mu}p^{\lambda} / m^2}{-p^2 + m^2}.
\end{align}
This momentum-space Green's function has two poles at $p^{0}= \pm E_p = \pm  \sqrt{\textbf{p}^2+m^2}
$. Time-ordering is taken care of with the $i\varepsilon$ prescription: 
\begin{align}
    \boxed{
        G^{\mu\lambda} =- i \frac{g^{\mu\lambda} - p^{\mu}p^{\lambda} / m^2}{p^2 - m^2+
        i\varepsilon}.
        }
\end{align}
Indeed the denominator can be expanded as
\begin{align}
     (p^{0})^2 - E_p^2+ i\varepsilon &= (p^{0})^2 - (E_p -
i\varepsilon)^2 \nonumber \\ &= (p^{0}-E_p +i\varepsilon) (p^{0}+E_p - i\varepsilon)
\end{align}
which means that I have moved the $+E_p$ pole down and the $-E_p$ pole up. Integration in
$\mathop{dp^{0}} e^{-ip^{0}(t-t^\prime)}$ is performed by closing the integration path in the upper
(lower) plane for $t<t^\prime$ ($t>t^\prime$), as per Jordan's lemma. So for $t<t^\prime$
($t>t^\prime$),
I pick up the $-E_p$ ($+E_p$) pole. This is an equivalent
prescription to the Feynman integration contour we have seen in class, and thus the boxed
quantity is the (momentum space) Feynman propagator.

\section{Exercise 4}%
\begin{exercise}
   Compute the unpolarized, differential cross-section in the center of mass frame
for the annihilation of one electron and one positron into two photons at leading
order in QED:
\begin{align}
    e^{+}e^{-}\to \gamma\gamma.
\end{align}
\end{exercise}
\subsection{Initial and final particle states, Feynman amplitudes}%
The in and out states are the direct product of single-particle states:
\begin{align}
    \ket{\Psi_\alpha} &= \ket{e^{-}(p_1,s) e^{+}(p_2,q)} = \sqrt{2E_{p_1}2E_{p_2}}
    c^\dagger (p_1,s) d^\dagger (p_2,q) \ket{0} \nonumber \\ 
    \ket{\Psi_\beta} &= \ket{\gamma (k_1,\lambda_1) \gamma(k_2,\lambda_2)} = \sqrt{2E_{k_1}2E_{k_2}}
    a^\dagger (k_1,\lambda_1) a^\dagger (k_2,\lambda_2) \ket{0} .
\end{align}
Obviously the diagrams will have to be at least second-order, since vertices have 3 legs
and we have 4 players. The (LO) amplitude is thus given by:
\begin{align}
i\mathcal{M}= 
\underbrace{\begin{gathered}
                \feynmandiagram [vertical=a to b] {
                i1 [particle=\(u^s (p_1)\)] -- [fermion] a --
                [photon] f1[particle = \(\varepsilon_{\lambda_1}(k_1)\)],
                i2 [particle=\(\bar{v}^{q}(p_2)\)] -- [anti fermion] b -- [photon]
                f2[particle=\(\varepsilon_{\lambda_2}(k_2)\)],
                a -- [fermion, momentum = {\(q\)}] b,
                };  
                    \end{gathered}}_{i\mathcal{M}_t} +
                \underbrace{    \begin{gathered}
                            \feynmandiagram[horizontal=i1 to i2,remember picture]{
                            f1[particle=\(u^{s}(p_1)\)]--[fermion,]b--[photon,
                            opacity=0]f2[particle=\(\varepsilon_{\lambda_1(k_1)}\)],
                            i1[particle=\(\bar{v}^{q}(p_2)\)]--[anti fermion,]a--[photon,opacity=0]i2[particle=\(\varepsilon_{\lambda_1(k_1)}\)],
                            b--[fermion, momentum'=\(\tilde{q}\)]a,
                            };
                            \begin{tikzpicture}[overlay,remember picture]
                            \begin{feynman}
                            \path (b) -- (i2) coordinate[midway] (b1)
                            (a) -- (f2) coordinate[midway] (a1);
                            \diagram*{
                            (b) --[photon] (b1) -- [photon] (i2),
                            (a) --[photon,rubout] (a1) -- [photon,] (f2)
                            };
                            \end{feynman}
                            \end{tikzpicture}
                            \end{gathered}}_{i\mathcal{M}_u}
.
\end{align}
I proceed with the evaluation of the diagrams. Going upstream as usual,
\begin{align}
i    \mathcal{M}_t &= \bar{v}^{q}(p_2) (-ie\gamma^{\mu})
    \frac{i\left(\slashed{q}+m\right)}{q^2-m^2 +i\varepsilon} (-ie \gamma^{\nu} )u^{s}(p_1)
    \varepsilon^\star_\mu (k_2,\lambda_2) \varepsilon^\star_\nu (k_1,\lambda_1),\nonumber \\ 
i    \mathcal{M}_u &= \bar{v}^{q}(p_2) (-ie\gamma^{\mu})
    \frac{i\left(\slashed{\tilde{q}}+m\right)}{\tilde{q}^2-m^2 +i\varepsilon} (-ie \gamma^{\nu} )u^{s}(p_1)
    \varepsilon^\star_\mu (k_1,\lambda_1) \varepsilon^\star_\nu (k_2,\lambda_2).
    \label{4.113}
\end{align}
From 4-momentum conservation: in $\mathcal{M}_t$, $q = p_1 - k_1 = k_2-p_2$; in $\mathcal{M}_u, $
$\tilde{q}
= p_1 - k_2 = k_1- p_2$. Thus the two diagrams become
\begin{align}
i    \mathcal{M}_t &= \frac{-ie ^2}{(p_1-k_1)^2 - m^2 + i\varepsilon}
    \bar{v}^{q}(p_2)\slashed{\varepsilon}^\star (k_2,\lambda_2)\left(\slashed{p}_1 -
    \slashed{k}_1 + m\right)\slashed{\varepsilon}^\star(k_1,p_1) u^{s}(p_1)\nonumber \\ 
 i   \mathcal{M}_u &= \frac{-ie ^2}{(p_1-k_2)^2 - m^2 + i\varepsilon}
    \bar{v}^{q}(p_2)\slashed{\varepsilon}^\star (k_1,\lambda_1)\left(\slashed{p}_1 -
    \slashed{k}_2 + m\right)\slashed{\varepsilon}^\star(k_2,p_2) u^{s}(p_1), 
\end{align}
where $\slashed{\varepsilon}^\star = \gamma^{\mu} \varepsilon_\mu $.

\subsection{Square matrix element calculation}%
\begin{definition}[Mandelstam variables]
    Whenever possible I will make the substitutions:
    \begin{align}
       s &= (p_1+p_2)^2 = (k_1+k_2)^2\nonumber \\  
       t &= (p_1-k_1)^2 = (p_2-k_2)^2 \nonumber \\ 
       u &= (p_1-k_2)^2 = (p_2-k_1)^2 
    \end{align}
\end{definition}
\begin{rmk}
    I list here some useful properties of the Mandelstam variables:
    \begin{align}
        s &= (p_1+p_2)^2 = 2m^2 + p_1\cdot p_2  = (k_1+k_2)^2 = k_1\cdot k_2\nonumber \\ 
        t &= (p_1-k_1)^2 = m^2 -p_1\cdot k_1  = (p_2-k_2)^2 = m^2 - k_2\cdot k_2\nonumber \\ 
        u &= (p_1-k_2)^2 = m^2- p_1\cdot k_2  = (p_2-k_1)^2 = m^2 - p_2\cdot k_1
        \label{4.116b}
    \end{align}
    and also
    \begin{align}
        t + s + u &= 2m^2    \label{4.117}
        \end{align}
\end{rmk}
\begin{proof}
The only non-trivial one is:
   \begin{align}
t+s+u &=  (p_1+p_2)^2 + (p_1-k_1)^2 + (p_1-k_2)^2 \nonumber \\ 
       &= 2m^2 + 2p_1\cdot p_2 + m^2 - 2p_1\cdot k_1 + m^2 - 2p_1\cdot k_2  
   \end{align} 
   and since ($p_1 + p_2 = k_1 + k_2,$) $k_1+k_2-p_2 = p_1$, the dotted products give
   $-2p_1\cdot p_1 = -2m^2$.
\end{proof}
When we take the square amplitude we have 4 terms: 
\begin{align}
    \left|\mathcal{M}\right|^2 &= \left|\mathcal{M}_t\right|^2 +
    \left|\mathcal{M}_u\right|^2 + (\mathcal{M}_t^\star \mathcal{M}_u + \mathcal{M}_t
    \mathcal{M}_u^\star)\nonumber \\ 
    &= \left|\mathcal{M}_t\right|^2 +
    \left|\mathcal{M}_u\right|^2 +2 \mathfrak{Re}(\mathcal{M}_t
    \mathcal{M}_u^\star)
\end{align}
If we ignore spin and helicity, we ought to sum on spin states and polarisation states. In
particular, polarisation sums give a nice result.
\begin{rmk}[Photon polarisation sums]
There are two physical polarisations. I can define them as vectors orthogonal to
$p^{\mu}$, and to another null vector $r^{\mu}$ (not proportional to $p^{\mu}$). If I
choose $r^{\mu} \coloneqq \bar{p}^{\mu}= (E_p, -\textbf{p})$, then the following holds:
\begin{align}
    \sum _\lambda \left(\varepsilon_\lambda\right)_\mu (p)
    \left(\varepsilon_\lambda^\star\right)_\nu(p) = -g_{\mu\nu} + \frac{p_\mu \bar{p}_\nu
    + p_\nu \bar{p}_\mu}{2E_p}\label{4.116}.
\end{align}
\end{rmk}
\begin{proof}
Like the polarisation sum in the previous exercise, this is proven in one specific frame (e.g. $p^{\mu}= (E_p , 0 ,0,p_z)$) and then written
in Lorentz-covariant form. I pick the usual normalisation $\varepsilon_\lambda \cdot
\varepsilon_\lambda =  -1$. Then the two physical polarisations are
\begin{align}
    \varepsilon_1 &= (0,1,0,0)\nonumber \\ 
    \varepsilon_2 &= (0,0,1,0).
\end{align}
Then 
\begin{align}
    \sum _\lambda \left(\varepsilon_\lambda\right)_\mu (p)
    \left(\varepsilon_\lambda^\star\right)_\nu(p) &= \begin{bmatrix} 0 \\1\\0\\0
    \end{bmatrix} \begin{bmatrix} 0&1&0&0 \end{bmatrix} +  \begin{bmatrix} 0 \\0\\1\\0
    \end{bmatrix} \begin{bmatrix} 0&0&1&0 \end{bmatrix}  \nonumber \\ 
    &= \begin{bmatrix} 0 & 0&0&0\\0&1&0&0\\0&0&1&0\\0&0&0&0 \end{bmatrix} .
\end{align}
In this frame, the RHS of \eqref{4.116} is 
\begin{align}
    \begin{bmatrix} -1 &0&0&0\\0&1&0&0\\0&0&1&0\\0&0&0&1 \end{bmatrix}  +
    \frac{1}{2E_p}\begin{bmatrix} E_p^2 +E_p^2 &&&-E_p p_z + E_p p_z\\&0&&\\&&0&\\E_p
    p_z - E_p p_z&&&-p_z^2-p_z^2 \end{bmatrix} 
\end{align}
and since $E_p^2- p_z^2=0$, the 00 and 33 entries of the RHS go to zero, which proves the
result in one frame. Since \eqref{4.116} is Lorentz-covariant, it must hold in all frames.
\end{proof}
\begin{lemma}
Due to the Ward identity, we can substitute
\begin{align}
\sum _\lambda (\varepsilon_\lambda)_\mu (\varepsilon_\lambda^\star)_\nu \to -g_{\mu\nu}
\end{align}
when summing over physical polarisations for our matrix elements.
\end{lemma}
\begin{proof}
We have terms of the type 
\begin{align}
\sum_{\text{pol}}\left|\mathcal{M}\right|^2 & = \sum_\lambda (\varepsilon_\lambda^\star)_\mu
\left( \mathcal{M}^\star \right) ^{\mu} (\varepsilon^{\lambda})_\nu \mathcal{M}^{\nu}\nonumber \\ 
&= \left(-g_{\mu\nu}+ \frac{p_\mu \bar{p}_\nu + p_\nu
\bar{p}_\mu}{2E_p}\right)\left(\mathcal{M}^\star\right)^{\mu}\mathcal{M}^{\nu}
\end{align}
and the second piece in brackets is zero due to Ward's identity ($p_{\mu}\mathcal{M}^\mu =
0$).\footnote{This is also the reason why I was free to pick $r^{\mu} = \bar{p}^{\mu}$
without consequences. Any other choice for $r^{\mu}$ would be a linear combination of my
choice and $p^{\mu}$, but the latter term would give zero at this level.}
\end{proof}
\subsubsection{Trace 1/3}%
I go on computing the first term, $\left|\overline{\mathcal{M}}_t\right|^2 $. As a
refresher for myself:
\begin{rmk}
    The h.c. of our spinor products is naturally
    \begin{align}
        \left(\bar{v}^{q} \gamma^{\mu} (\slashed{q})\gamma^{\nu} u^{s}\right)^\dagger =
        \bar{u}^{s} \gamma^{\nu} (\slashed{q}) \gamma^{\mu} v^{q}.
    \end{align}
\end{rmk}
\begin{proof}
Relies on $(\gamma^{\mu})^\dagger  = \gamma^{0}\gamma^{\mu}\gamma^{0}$ (alias $\gamma^0$
is hermitian and the others are antihermitian) and $\gamma^{0} \gamma^{0} = 1$:
    \begin{align}
        \left(\bar{v}^{q} \gamma^{\mu} (\slashed{q})\gamma^{\nu} u^{s}\right)^\dagger &=
        \left( u^{s} \right) ^\dagger\left(  \gamma^{0}\gamma^{\nu}\cancel{\gamma^{0}} \right)
        \left( \cancel{\gamma^{0}}\gamma^{\rho} \cancel{\gamma^{0}} \right)         q_\rho
        \left( \cancel{\gamma^{0}}\gamma^{\mu} \gamma^{0} \right)  \left[ \left(v^{q} \right)
        ^\dagger 
        \gamma^{0}\right]^\dagger \nonumber \\ 
        &= \bar{u}^{s} \gamma^{\nu} \gamma^{\rho} q_\rho\gamma^{\mu}\cancel{ \gamma^{0}}
     \cancel{   \gamma^{0}}v^{q}.
    \end{align}
\end{proof}
Similar things happen with the $m$ terms. Then
\begin{align}
    \left|\overline{\mathcal{M}}_t\right|^2 & \coloneqq \frac{1}{4}\sum_{\lambda,s,q}
    \left|\mathcal{M}_t\right|^2 \nonumber \\ 
    &= \frac{e^{4}}{4(t-m^2)^2}\sum_{\lambda,s,q} \bar{v}^{q}(p_2)
    \varepsilon_\mu^\star(\lambda_2,p_2) \gamma^{\mu} (\slashed{p}_1 - \slashed{k}_1 +m)
    \varepsilon_{\nu}^\star (\lambda_1,p_1) \gamma^{\nu} u^{s}(p_1)\times \nonumber \\ 
    &\phantom{=\frac{e^{4}}{(t-m^2)^2}\sum}\times \bar{u}^{s}(p_1)
    \gamma^{\nu^\prime}\varepsilon_{\nu^\prime}(\lambda_1,p_1) (\slashed{p}_1 -
    \slashed{k}_1 +m) \gamma^{\mu^\prime} \varepsilon_{\mu^\prime} (\lambda_2, p_2)
    v^{q}(p_2)
\end{align}
For clarity's sake I write the indices in spinor space:
\begin{align}
    \left|\overline{\mathcal{M}}_t\right|^2 
    &= \frac{e^{4}}{4(t-m^2)^2}\sum_{\mathbin{\color{brown}\lambda,\lambda^\prime},\mathbin{\color{red}s},\mathbin{\color{blue}q}} \mathbin{\color{blue}\left[ \bar{v}^{q}(p_2)
 \right] _{\alpha} }   \left[ \gamma^{\mu} \right] ^{\alpha\beta} \left[ (\slashed{p}_1 - \slashed{k}_1 +m)
 \right]_{\beta\gamma}\left[  \gamma^{\nu} \right] ^{\gamma\delta}\mathbin{\color{red}\left[  u^{s}(p_1)
 \right]_{\delta}} \times \nonumber \\ 
    &\phantom{=\frac{e^{4}}{(t-m^2)^2}\sum}\times\mathbin{\color{red} \left[ \bar{u}^{s}(p_1)
 \right]_\varepsilon} \left[ \gamma^{\nu^\prime} \right]^{\varepsilon\zeta} \left( (\slashed{p}_1 -
    \slashed{k}_1 +m) \right)_{\zeta\eta} \left[ \gamma^{\mu^\prime}  \right]
    ^{\eta\theta}\mathbin{\color{blue}\left[   v^{q}(p_2) \right] _\theta }\times  \nonumber \\
&\phantom{=\frac{e^{4}}{(t-m^2)^2}\sum}\times \mathbin{\color{brown}\varepsilon_\mu^\star(\lambda_2,p_2)
\varepsilon_{\mu^\prime} (\lambda_2, p_2)\varepsilon_{\nu^\prime}(\lambda_1,p_1)
\varepsilon_{\nu}^\star (\lambda_1,p_1) }.
\end{align}
I can break down the coloured pieces: spinor sums satisfy the completeness relations,
\begin{align}
\mathbin{\color{blue}\sum _q \bar{v}^{q}(p_2)_\alpha v^{q}(p_2)_\theta = \left( \slashed{p}_2 - m \right) _{\alpha\theta}}
\end{align}
and
\begin{align}
\mathbin{\color{red}\sum _q u^{s}(p_1)_\delta \bar{u}^{s}(p_1)_\varepsilon = \left(
\slashed{p}_1+ m \right) _{\delta\varepsilon}}.
\end{align}
The terms in brown simply become 
\begin{align}
    \mathbin{\color{brown}g_{\mu\mu^\prime} g_{\nu\nu^\prime}}.
\end{align}
I drop the indices now. Since they are all contracted, the matrix element is a trace of gamma matrices:
\begin{align}
    \left|\overline{\mathcal{M}}_t\right|^2 &=
    \frac{e^{4}}{4(t-m^2)^2}\Tr\left\{(\slashed{p}_2-m) \gamma^{\mu}
    (\slashed{p}-\slashed{k}_1 +m) \gamma^{\nu}(\slashed{p}_1+m)\gamma_\nu (\slashed{p}-\slashed{k}_1 +m)
    \gamma_\mu\right\}.
\end{align}
This calls for some manipulation. First, trace is invariant under cyclic permutations. So
I move the last $\gamma_\mu$ to the left. Then I use this:
\begin{rmk} 
    \begin{align}
        \gamma_\mu (\slashed{p}_2 \pm  m) \gamma^{\mu} = -2\slashed{p}_2 \pm 4m.\label{gronz}
    \end{align}
\end{rmk}
\begin{proof}
    By direct computation.
    \begin{align}
        \gamma_\mu (\slashed{p} \pm  m) \gamma^{\mu} &= \gamma_\mu (\gamma^{\nu}p_\nu \pm  m)
        \gamma^{\mu}\nonumber \\ 
        &= \gamma_\mu \gamma^{\nu}p_\nu \gamma^{\mu} \pm  m\gamma_\mu\gamma^{\mu}.
        \\ \intertext{
    The Clifford algebra on the first term gives $\gamma_\mu \gamma^{\nu} \gamma^{\mu} =
    -\gamma_\mu \gamma^{\mu}\gamma^{\nu} + 2\gamma_\mu g^{\mu\nu}$, whereas the second
    term has a $\gamma_\mu \gamma^{\mu}=  4$:}
        & = -4 \gamma^{\nu}p_\nu + 2\gamma^{\nu}p_\nu - 4 m  = -2\gamma^{\nu}p_\nu \pm 4m.
    \end{align}
I group a 4 outside:
\end{proof}
\begin{align}
    \left|\overline{\mathcal{M}}_t\right|^2 &=
    \frac{e^{4}}{(t-m^2)^2}\Tr\left\{\left(\slashed{p}_2+2m\right)
    (\slashed{p}_1-\slashed{k}_1 +m) (\slashed{p}_1-2m) (\slashed{p}_1-\slashed{k}_1 +m)\right\}.
\end{align}
The next step is to break down this trace. Due to the Clifford algebra, an odd number of
$\gamma$ matrices is traceless. Then
\begin{align}
   \Tr\left\{\ldots\right\} &= \Tr \left\{\slashed{p}_2
    (\slashed{p}_1-\slashed{k}_1)
    \slashed{p}_1 (\slashed{p}_1- \slashed{k}_1)\right\} + \Tr\left\{\slashed{p}_2
    (\slashed{p}_1 - \slashed{k}_1) (-2m) m\right\} + \Tr\left\{\slashed{p}_2 m
    \slashed{p}_1 m  \right\}  +\nonumber \\ 
    &+\Tr\left\{\slashed{p}_2 m (-2m) (\slashed{p}_1 - \slashed{k}_1)\right\}+ \Tr
   \left\{(2m) (\slashed{p}_1-\slashed{k}_1) \slashed{p}_1 m\right\} +\Tr\left\{ (2m)
   (\slashed{p}_1 - \slashed{k}_1) (-2m) (\slashed{p}_1-\slashed{k}_1)\right\} +\nonumber \\ 
   &+ \Tr\left\{(2m)m \slashed{p}_1 (\slashed{p}_1-\slashed{k}_1)\right\} +
   \Tr\left\{(2m)m(-2m)m\right\}\nonumber \\ 
   &= \Tr\left\{\slashed{p}_2 (\slashed{p}_1-\slashed{k}_1) \slashed{p}_1
   (\slashed{p}_1 - \slashed{k}_1)\right\} - 2m^2 \Tr\left\{\slashed{p}_2 (\slashed{p}_1 -
   \slashed{k}_1)\right\} + m^2 \Tr\left\{\slashed{p}_2\slashed{p}_1\right\} - 2m^2
   \Tr\left\{\slashed{p}_2(\slashed{p}_1 - \slashed{k}_1)\right\} +\nonumber \\ 
   &+ 2m^2 \Tr\left\{(\slashed{p}_1-\slashed{k}_1)\slashed{p}_1\right\} - 4m^2
   \Tr\left\{(\slashed{p}_1-\slashed{k}_1) (\slashed{p}_1-\slashed{k}_1)\right\} + 2m^2
   \Tr\left\{\slashed{p}_1 (\slashed{p}_1-\slashed{k}_1)\right\} - 4m^{4}\Tr \mathbb{I} \nonumber \\ 
   &= \Tr\left\{\slashed{p}_2 (\slashed{p}_1-\slashed{k}_1) \slashed{p}_1
   (\slashed{p}_1-\slashed{k}_1)\right\} - 4m^2 \Tr \left\{\slashed{p}_2 (\slashed{p}_1 -
   \slashed{k}_1)\right\} + m^2 \Tr \left\{\slashed{p}_2 \slashed{p}_1\right\}+\nonumber \\ &+ 4m^2
   \Tr\left\{(\slashed{p}_1- \slashed{k}_1)\slashed{p}_1\right\} - 4m^2 (p_1-k_1)^2 \Tr
   \mathbb{I} - 4m^{4}\Tr \mathbb{I}
\end{align}
The first term can be further simplifed, using $\slashed{p}_1 \slashed{p}_1 = p_1\cdot
p_1 = m^2$ and $\slashed{k}_1 \slashed{k}_1 = k_1\cdot k_1=  0$:
\begin{align}
   \slashed{p}_2 (\slashed{p}_1-\slashed{k}_1)\slashed{p}_1(\slashed{p}_1-\slashed{k}_1)
   &= \slashed{p}_2  (\slashed{p}_1-\slashed{k}_1)(m^2 - \slashed{p}_1 \slashed{k}_1) \nonumber \\ 
   &= \slashed{p}_2 (m^2 \slashed{p}_1 - m^2 \slashed{k}_1
   - m^2 \slashed{k}_1 + \slashed{k}_1 \slashed{p}_1 \slashed{k}_1)\nonumber \\ 
   &= m^2 \slashed{p}_2 \slashed{p}_1 -2 m^2 \slashed{p}_2 \slashed{k}_1 + \slashed{p}_2
   \slashed{k}_1 \slashed{p}_1 \slashed{k}_1.
\end{align}
There is no sugar-coating this. I exploit the Dirac trace
\begin{align}
    \Tr\left\{\slashed{a} \slashed{b}\right\} = 4 a\cdot b.
\end{align}
These types of traces can be found starting from the trace of gamma matrices, in this case
\begin{align}
    \Tr \left\{\gamma_\mu \gamma_\nu\right\} = 4g_{\mu\nu}, 
\end{align}
and making the formal substitution for the Clifford algebra:
\begin{align}
    \left\{\gamma_\mu, \gamma_\nu\right\} = 2 g_{\mu\nu}\quad \implies
    \left\{\slashed{a},\slashed{b}\right\}= 2a\cdot b.
\end{align}
Thanks to the Clifford algebra, the term with 4 slashed vectors cam be simplified as
\begin{align}
    \slashed{p}_2 \slashed{k}_1 \slashed{p}_1 \slashed{k}_1 = - \underbrace{\slashed{p}_2 \slashed{p}_1
    \slashed{k}_1 \slashed{k}_1 }_{0}+ 2\slashed{p}_2 \slashed{k}_1 (p_1\cdot k_1)
\end{align}
Thus all is left is traces of two slashed vectors:
\begin{align}
    \Tr\left\{\ldots\right\}&= 4(2(p_1\cdot k_1) (p_2\cdot k_1) + m^2 (p_2\cdot p_1 -
    2p_2\cdot k_1) +m^2 (-4 p_2(p_1-k_{1})
   +\nonumber \\ &+
    p_2\cdot p_1 + 4 (p_1-k_1)p_1 - 4(p_1-k_1)^2 )-4m^4)\nonumber \\ 
    &= 4 ( 2(p_1\cdot k_1)(p_2\cdot k_1) + m^2( 2 p_2\cdot p_1 -2 p_2\cdot k_1
    -4p_2\cdot p_1 +4p_2\cdot k_1 +\nonumber \\ 
    &+4m^2 - 4k_1\cdot p_1 -4 m^2 + 8 p_1\cdot k_1) - 4m^{4})\nonumber \\ 
    &= 8 ((p_1\cdot k_1)(p_2\cdot k_1) + m^2(-p_2\cdot p_1  + p_2\cdot k_1 + 2p_1\cdot
    k_1)- 2m^{4}).
    \end{align}
This can be conveniently expressed in terms of Mandelstam by substituting \eqref{4.116b}:
\begin{align}
    \Tr\left\{\ldots\right\}&= 8 \left(\frac{m^2-t}{2}\frac{m^2-u}{2} + m^2 2
    \frac{m^2-t}{2} + m^2 \frac{m^2-u}{2} - m^2 \frac{s}{2} + m^{4}- 2m^{4}\right).
\end{align}
I can get rid of $s = 2m^2 - u - t$ (from \eqref{4.117}): 
\begin{align}
    \Tr\left\{\ldots\right\}&= 8 \left(\frac{1}{4}(m^2-t)(m^2-u) + m^2(m^2-t) +
    \frac{1}{2}m^2 (m^2-u) - \frac{m^2}{2}(2m^2-t-u)- m^{4}\right)\nonumber \\ 
    &=2\left(m^2-t\right)\left(m^2-u\right) + 8 m^2(m^2-t) + 4 m^2(m^2-u) - 16m^{4} +
    4m^2t + 4m^2u\nonumber \\ 
    &= 2m^{4} - 2 m^2t - 2m^2u + 2ut + 8m^{4} - 8m^2t + 4m^{4} - 4m^2u -16m^{4} + 4m^2t +
    4m^2u\nonumber \\ 
    &=-6m^2t - 2m^2u +2ut -2m^{4}  
\end{align}
which can be factored into
\begin{align}
    \Tr\left\{1\right\} = 2(t-m^2)(u-3m^2) - 8m^{4}.
\end{align}

\subsubsection{Trace 2/3}%
The second trace would be that coming from $\left|\mathcal{M}_u\right|^2 $. Fortunately,
this one we can just read off the previous result: the amplitudes \eqref{4.113} are
related by the substitution $(k_1,\lambda_1) \leftrightarrow (k_2,\lambda_2)$, and
(subsequently) $q \leftrightarrow \tilde{q}$. The polarisation indices are dummies since I
am summing over them. However the momenta are
interchanged: instead of $q = p_1-k_1$, I have $\tilde{q} = p_1-k_2$. The square matrix
element will be the same as the previous one, except now $u\leftrightarrow t$.
\begin{align}
    \Tr\left\{2\right\} = 2(u-m^2)(t-3m^2) - 8m^{4}.
\end{align}
\subsubsection{Trace 3/3}%
The interference term is
\begin{align}
2\mathfrak{Re} \overline{\mathcal{M}}_t \overline{\mathcal{M}}_u^\star& \coloneqq
\frac{1}{4}\sum_{\lambda,s,q}  \overline{\mathcal{M}}_t \overline{\mathcal{M}}_u^\star
+\text{c.c.}\nonumber \\ 
    &= \frac{2e^{4}}{4(t-m^2)(u-m^2)}\sum_{\lambda,s,q} \bar{v}^{q}(p_2)
    \mathbin{\color{brown}\varepsilon_\mu^\star(\lambda_2,p_2)} \gamma^{\mu} (q+m)
    \mathbin{\color{brown}\varepsilon_{\nu}^\star (\lambda_1,p_1)} \gamma^{\nu} u^{s}(p_1)\times \nonumber \\ 
    &\phantom{=\frac{e^{4}}{(t-m^2)^2}\sum}\times \bar{u}^{s}(p_1)
    \gamma^{\nu^\prime}\mathbin{\color{brown}\varepsilon_{\nu^\prime}(\lambda_2,p_2)
    (}\tilde{\slashed{q}}+m) \gamma^{\mu^\prime} \mathbin{\color{brown}\varepsilon_{\mu^\prime} (\lambda_1, p_1)
}    v^{q}(p_2).
\end{align}
The difference here is that the polarisation sums in brown are mismatched, as now we get a
 \begin{align}
     g_{\mu\nu^\prime } g_{\mu^\prime\nu}
 \end{align}
 which means that it is no longer adjacent gamma matrices that are contracted together.
 With this in mind, I the trace to compute is
 \begin{align}
 \Tr\left\{3\right\}&=
    \frac{1}{4}\Tr\left\{(\slashed{p}_2-m) \gamma^{\nu}
    (\slashed{q}+m) \gamma^{\mu}(\slashed{p}_1+m)\gamma_\nu
    (\tilde{\slashed{q}}+m)
    \gamma_\mu\right\}\nonumber \\ 
    &= \frac{1}{4}\Tr\{\underbrace{\gamma^{\nu}(\slashed{q}+m) \gamma^{\mu}(\slashed{p}_1 +m ) \gamma_\nu
}_{A}    (\tilde{\slashed{q}}+m \gamma_\mu )(\slashed{p}_2-m)\}.
 \end{align}
 The terms I called A become
 \begin{align}
     A &= m^2 \gamma^{\nu}\gamma^{\mu} \gamma_\nu + m\gamma^{\nu}(\slashed{q}\gamma^{\mu}+
     \gamma^{\mu}\slashed{p}_1)\gamma_\nu  +
     \gamma^{\nu}\slashed{q}\gamma^{\mu}\slashed{p}_1\gamma_\nu .
 \end{align}
Here is some Dirac algebra:
 \begin{align}
\gamma^{\mu}   \gamma_\mu &= 4,& \gamma^{\mu}\slashed{a}\gamma_\mu &=
-2\slashed{a}, &\gamma^{\mu}\slashed{a}\slashed{b} \gamma_\mu &= 4a\cdot b, &
\gamma^{\mu}\slashed{a}\slashed{b}\slashed{c}\gamma_\mu &=
-2\slashed{c}\slashed{b}\slashed{a} \nonumber \\ 
&&\gamma^{\mu} \gamma^\nu \gamma_\mu  &=
 -2\gamma^{\nu}, & \gamma^{\mu}\gamma^{\alpha}\gamma^{\beta}\gamma_\mu &= 
 4g^{\alpha\beta},&\gamma^{\mu}\gamma^{\alpha}\gamma^{\beta}\gamma^{\rho}\gamma_\mu &=
 -2 \gamma^{\rho}\gamma^{\beta}\gamma^{\alpha}. \label{dalgebra}
 \end{align}
(The relations in the second line are standard; those in the first line are consequence of
the aforementioned substitution $g_{\mu\nu}\leftrightarrow \cdot $.)
I also employ a mixed result which is not as easy to see with this analogy, hence I prove it:
\begin{rmk}
   \begin{align}
       \gamma^{\nu}\slashed{a} \gamma^{\mu}\gamma_\nu = 4a^{\mu}.
   \end{align} 
\end{rmk}
\begin{proof}
   \begin{align}
       \gamma^{\nu} \slashed{a} \gamma^{\mu} \gamma_\nu &= a_\rho\gamma^{\nu}
       \gamma^{\rho}\gamma^{\mu} \gamma_\nu \nonumber \\ 
       &= a_\rho 4g^{\rho\mu} = 4a ^{\mu},
   \end{align} 
   where in the first line I am free to move $a_\rho$ as it is just a scalar to spinorial
   indices.
\end{proof}

Thus 
 \begin{align}
     A &= m^2 (-2\gamma^{\mu}) + 4m (q+p_1)^{\mu} -2 \slashed{p}_1\gamma^{\mu}\slashed{q}
 \end{align}
 and the trace is 
 \begin{align}
     \Tr\{3\} &= \frac{1}{4}\Tr\{[\underbrace{m^2 (-2\gamma^{\mu})}_{1}
     +\underbrace{ 4m (q+p_1)^{\mu}}_{0} -\underbrace{2 \slashed{p}_1\gamma^{\mu}\slashed{q}
}_{3}]\times (\tilde{\slashed{q}}+m)\underbrace{ \gamma_\mu
}_{1}(\slashed{p}_2-m)\}.
 \end{align}
 The underbraces signify how many gamma matrices each term has. We want an even number of
 them and thus
 \begin{align}
     \Tr\left\{3\right\} &= \frac{1}{4}\Tr \left\{ \left[ -2m^2 \gamma^{\mu}-
     2\slashed{p}_1\gamma^{\mu}\slashed{q} \right] ( -m^2\gamma_\mu +
     \tilde{\slashed{q}}\gamma_\mu\slashed{p}_2)\right\} + \nonumber \\ 
     & +\frac{1}{4} \Tr\left\{4m^2 (q+p_1)^{\mu} \left(-\tilde{\slashed{q}}\gamma_\mu + \gamma_\mu
     \slashed{p}_2\right)\right\}.
 \end{align}
 The first term gives (I use \eqref{dalgebra} in the second line)
 \begin{align}
     \Tr\left\{3.1\right\} &= 2 \left\{m^4 \Tr\left[\gamma^{\mu}\gamma_\mu\right] -
     m^2\Tr\left[\gamma^{\mu}\tilde{\slashed{q}}\gamma_\mu \slashed{p}_2\right] + m^2
     \Tr\left[\slashed{p}_1 \gamma^{\mu}\slashed{q}\gamma_\mu \right] -
     \Tr\left[\slashed{p}_1 \gamma^{\mu}\slashed{q}
     \tilde{\slashed{q}}\gamma_\mu \slashed{p}_2\right]\right\}\nonumber \\ 
     &= 2 \left\{16m^{4}  - m^2 \Tr\left[-2\tilde{\slashed{q}}\slashed{p}_2\right] + m^2
     \Tr\left[\slashed{p}_1 (-2\slashed{q})\right]- \Tr\left[\slashed{p}_1 4\left( q\cdot
     \tilde{q} \right)  \slashed{p}_2\right]\right\}\nonumber \\ 
     &= 16 \left\{2m^{4} + m^2 \tilde{q}\cdot p_2 -m^2 p_1\cdot q - 2(q\cdot \tilde{q})(p_1\cdot
     p_2)  \right\} ;
 \end{align}
 the second term gives 
 \begin{align}
     \Tr\left\{3.2\right\} &= 4m^2 (q+p_1)^{\mu}\Tr\left[\tilde{\slashed{q}}\gamma_\mu -
     \gamma_\mu \slashed{p}_2\right]\nonumber \\ 
     &= 4m^2 (q+p_1)^{\mu}\Tr\left(\gamma_\mu ( \slashed{p}_2)-\tilde{\slashed{q}}\right) \nonumber \\ 
     &= 16m^2 (q+p_1)^{\mu} (p_2- \tilde{q})_\mu.
 \end{align}
 Summing the contributions gives
 \begin{align}
     \Tr\left\{3\right\}&= 4 \left( 2 m^{4 }  + m^2 \tilde{q}\cdot p_2 - m^2 q\cdot p_1 -
     2 \left(q\cdot \tilde{q}\right)\left(p_1\cdot p_2\right) + (q+p_1)\cdot
     (p_2-\tilde{q})\right)\nonumber \\ 
     &= -8 (q\cdot \tilde{q})(p_1\cdot p_2) + 4m^2(\tilde{q}p_2 - qp_1 + qp_2 - q\tilde{q} 
     +p_1p_2 - p_1\tilde{q}) +8m^{4}\nonumber \\ 
     &= -8 (q\tilde{q})(p_1p_2) + 4m^2 [ (q+\tilde{q})(p_2-p_1) - q\tilde{q} + p_1p_2 ] +
     8m^{4}.
     \end{align}
     The first term in square brackets simplifies to (using $\tilde{q} = p_1-k_2 =
     k_1-p_2$)
     \begin{align}
         (q+\tilde{q}) (p_2-p_1)  = (p_1-\cancel{k_1 }+\cancel{ k_1}- p_2)(p_2-p_1) = -2
         m^2 + 2p_1\cdot p_2;
     \end{align}
     the term with $q\tilde{q}$ become (using $k_1+k_2 = p_1+p_2$)
     \begin{align}
         q\tilde{q} = (p_1-k_1) (p_1-k_2) &= p_1^2 - p_1(k_1+k_2) + k_1k_2\nonumber \\ 
         &=p_1^2 - p_1(p_1+p_2) +k_1k_2 \nonumber \\ 
         &= k_1k_2- p_1p_2 = \frac{1}{2}(k_1+k_2)^2 - \frac{1}{2} (p_1+p_2)^2 + m^2 = m^2.
     \end{align}
     Thus
     \begin{align}
         \Tr\left\{3\right\}&= -8m^2 p_1p_2 +4m^2 (-2m^2 +2p_1p_2 - m^2 + p_1p_2)
         +8m^{4}\nonumber \\ 
         &= 4m^2 p_1p_2 -4m^2 .
     \end{align}
     Now I can introduce the Mandelstam variables: $p_1 p_2 = (s-2m^2) /2 $ from
     \eqref{4.116b}:
     \begin{align}
        \Tr\left\{3\right\}&= 2m^2 s - 4m^2 - 4m^2 = 2m^2 s - 8m^2.
     \end{align}
     Since the other traces are in terms of $u,t$, I also employ \eqref{4.117} ($t+u+s =
     2m^2$),
     \begin{align}
         \Tr\left\{3\right\}&= -2m^2 t - 2m^2 u - 4m^{4}\nonumber \\ 
         &= 2 m^2 ( m^2- t) + 2m^2 (m^2-u) -8m^{4}.
     \end{align}
     By the way this one piece is real, so the interference term will just be twice this.
\subsubsection{Square matrix element}%
All is left is to recover the previous results: 
\begin{align}
    \Tr\left\{1\right\} &= 2(t-m^2)(u-3m^2) - 8m^{4}\nonumber \\ 
    \Tr\left\{2\right\} &= 2(u-m^2)(t-3m^2) - 8m^{4}\nonumber \\ 
    \Tr\left\{3\right\} &=  -2m^2 (t-m^2) - 2m^2(u-m^2) - 8m^{4}.
\end{align}
The square matrix element has some further factors:
\begin{align}
    \left|\overline{\mathcal{M}}\right|^2 &= \frac{e^{4}}{(t-m^2)^2}\Tr\left\{1\right\} +
    \frac{e^{4}}{(u-m^2)^2} \Tr{2} + \frac{2e^{4}}{(t-m^2)(u-m^2)}\Tr{3}\nonumber \\ 
    &= 2e^{4} \Bigg\{ \frac{u-3m^2}{t-m^2} - \frac{4m^{4} }{(t-m^2)^2} +
    \frac{t-3m^2}{u-m^2} - \frac{4m^{4}}{(u-m^2)^2} + \nonumber \\ 
    &\phantom{=2e^{4}\Bigg\{}- \frac{2m^2}{u-m^2} - \frac{2m^2}{t-m^2}
 -  \frac{8m^{4}}{(u-m^2)(t-m^2)} \Bigg\}\nonumber \\ 
&= 2e^{4}\Bigg\{ \frac{u-m^2}{t-m^2} - \frac{4m^2}{t-m^2} + \frac{t-m^2}{u-m^2} -
\frac{4m^2}{u-m^2} + \nonumber \\ &\phantom{=2e^{4}\Bigg\{}- 4m^4
\left(\frac{1}{(t-m^2)^2}+\frac{1}{(u-m^2)^2}+\frac{2}{(u-m^2)(t-m^2)}\right) \Bigg\}.
\end{align}
The terms in brackets are the square of a binomial:
\begin{align}
    \left|\overline{\mathcal{M}}\right|^2 &= 2e^{4}\left\{\frac{u-m^2}{t-m^2} +
    \frac{t-m^2}{u-m^2}- \frac{4m^2}{t-m^2}- \frac{4m^2}{u-m^2} -
    4m^4\left(\frac{1}{t-m^2}+ \frac{1}{u-m^2}\right)^2\right\}\nonumber \\ 
&= 2e^{4}\left\{\frac{u-m^2}{t-m^2} +
    \frac{t-m^2}{u-m^2}- 4m^2\left( \frac{1}{t-m^2}- \frac{1}{u-m^2}  \right) -
    4m^4\left(\frac{1}{t-m^2}+ \frac{1}{u-m^2}\right)^2\right\}.
\end{align}
Inserting a $+1-1$, and using the $-1$ to complete a square with the last two terms yields
my final result:
 \begin{align}
\boxed{
    \left|\overline{\mathcal{M}}\right|^2 = 2e^{4}\left\{\frac{u-m^2}{t-m^2} +
    \frac{t-m^2}{u-m^2} + 1 - \left[1 +
    2m^2\left(\frac{1}{t-m^2}+\frac{1}{u-m^2}\right)\right]^2\right\}.
    }
\end{align}
\subsection{Differential cross section in CM frame}%
\subsubsection{Kinematics}%
For the cross section calculation, I start with some kinematics. In the CM frame,
\begin{align}
   p^{\mu} _{1,2} &= (E, \pm  \textbf{p}), & E &= (\textbf{p}^2 +m^2)^{1 /2},\nonumber \\ 
    k_{1,2}^{\mu} &= (\omega, \pm \textbf{k}) ,& \omega &= \textbf{k}^2 = E.
\end{align}
The Mandelstam variable simplify to (defining $\cos\theta = \textbf{p}\cdot \textbf{k}$):
\begin{align}
    s &= (p_1+p_1)^2 = (2E^2) = 4E^2 \nonumber \\ 
    t&= (p_1-k_1)^2 = - (\textbf{p} - \textbf{k})^2 = -\textbf{p}^2 - E^2 +
    2 \left|\textbf{p}\right| E \cos\theta\nonumber \\ 
    u &= (p_1-k_2)^2 = - (\textbf{p}+\textbf{k})^2 = -\textbf{p}^2 - E^2 - 2
    \left|\textbf{p}\right|E\cos\theta .
\end{align}
\subsubsection{Cross section calculation}%
I unwind $\left|\overline{\mathcal{M}}\right|^2 $ piece by piece:
\begin{align}
    t-m^2 &= \underbrace{- \textbf{p}^2 - m^2}_{-E^2} - E^2 + 2 \left|p\right|E\cos\theta \nonumber \\ 
    &= -2E (E- \left|\textbf{p}\right|\cos\theta);
\end{align}
likewise
\begin{align}
    u-m^2 = -2E (E + \left|\textbf{p}\right|\cos\theta).
\end{align}
Thus the first three terms inside curly braces become
\begin{align}
    \frac{u-m^2}{t-m^2} + \frac{t-m^2}{u-m^2} + 1 &=
    \frac{E+\left|\textbf{p}\right|\cos\theta}{E-\left|\textbf{p}\right|\cos\theta} +
    \frac{E-\left|\textbf{p}\right|\cos\theta}{E + \left|\textbf{p}\right|\cos\theta}+1 \nonumber \\ 
    &= \frac{(E+\left|\textbf{p}\right|\cos\theta)^2 +
    (E-\left|\textbf{p}\right|\cos\theta)^2 + (E+\left|\textbf{p}\right|\cos\theta
    )(E-\left|\textbf{p}\right|\cos\theta)}{E^2 - \textbf{p}^2 \cos^2\theta}\nonumber \\ 
    &= \frac{2E^2 + 2\textbf{p}^2\cos^2\theta + E^2 - \textbf{p}^2 \cos\theta}{E^2-
    \textbf{p}^2 \cos^2\theta} \nonumber \\ 
    &= \frac{3E^2 + \textbf{p}^2 \cos^2\theta}{E^2- \textbf{p}^2\cos^2\theta}\\
    \intertext{
and I substitute $E^2 = \textbf{p}^2 + m^2$:}
    & = \frac{3m^2 + \textbf{p}^2 (3 + \cos^2\theta)}{m^2 + \textbf{p}^2(1-\cos^2\theta)}
    \nonumber \\ 
    & = \frac{3m^2 + \textbf{p}^2 (3 + \cos^2\theta)}{m^2 + \textbf{p}^2\sin^2\theta}.
\end{align}
Next up the last two terms are
\begin{align}
    2m^2\left(  
    \frac{1}{t-m^2} + \frac{1}{u-m^2}\right)  &= 2m^2 \left(\frac{1}{-2E(E -
    \left|p\right|\cos\theta)} + \frac{1}{-2E(E+\left|p\right|\cos\theta)}\right).\\ \intertext{
The denominator is the same as above; upon the same $E$ subsitution, it becomes}
    & = -\frac{m^2}{E}\left(\frac{E + \cancel{\left|p\right|\cos\theta }+ E -
    \cancel{\left|p\right|\cos\theta}}{m^2 + \textbf{p}^2 \sin^2\theta}\right)\nonumber \\ 
    &= - \frac{2m^2}{m^2 + \textbf{p}^2 \sin^2\theta}\nonumber \\ 
    1 + 2m^2\left(  
    \frac{1}{t-m^2} + \frac{1}{u-m^2}\right)  &= 1-\frac{2m^2}{m^2 + \textbf{p} ^2\sin^2\theta}\nonumber \\ 
    &= \frac{\textbf{p}^2 \sin^2\theta-m^2}{\textbf{p}^2 \sin^2\theta+m^2}
\end{align}
which we then square and join in with the other piece:
\begin{align}
   \left|\overline{\mathcal{M}}\right|^2   &= 2e^{4}\left\{\frac{3m^2 +
   \textbf{p}^2(3+\cos^2\theta)}{m^2 + \textbf{p}^2\sin^2\theta} -
   \left(\frac{\textbf{p}^2\sin^2\theta-m^2}{\textbf{p}^2\sin^2\theta+m^2}\right)^2\right\}.
\end{align}
From the textbooks, we find an expression for the CM differential cross-section:
\begin{align}
    \left(\frac{d\sigma}{d\Omega}\right)_\text{cm}=
    \frac{\left|\textbf{k}\right|}{\left|\textbf{p}\right|}\frac{\left|\overline{\mathcal{M}}\right|^2 }{64 
    \pi^2 s}& = \frac{2e^{4}}{64\pi^2 s}
    \frac{E}{\left|\textbf{p}\right|}\frac{1}{4E^{2}}\left\{\ldots\right\} .
    \end{align}
I introduce an $\alpha = e^2 / \left( 4\pi \right) $:
\begin{align}
\boxed{
    \left(\frac{d\sigma}{d\Omega}\right)_\text{cm} =
        \frac{\alpha^2}{8}\frac{1}{E \left|\textbf{p}\right|}\left\{\frac{3m^2 +
        \textbf{p}^2(3+\cos^2\theta)}{m^2+\textbf{p}^2\sin^2\theta} -
        \left(\frac{\textbf{p}^2\sin^2\theta-m^2}{\textbf{p}^2\sin^2\theta+m^2}\right)^2\right\},
    }
\end{align}
\subsection{NR limit}%
In the NR limit ($p^2 \ll m^2 \approx E^2$), the terms inside curly braces simplify to
\begin{align}
    \left\{\ldots\right\}\overset{\text{nr}}{\longrightarrow} \frac{3m^2}{m^2} - \left( \frac{m^2}{m^2} \right) ^2 = 2
\end{align}
and the differential cross section becomes
\begin{align}
    \left(\frac{d\sigma}{d\Omega}\right)_\text{cm,NR} = \frac{\alpha^2}{4} \frac{1}{m
    \left|\textbf{p}\right|}.
\end{align}
I integrate over the solid angle, which is 
\begin{align}
   \Omega = \frac{4\pi}{2}  = 2\pi
\end{align}
since the two final particles are indistinguishable (this prevents the state $(k_1,k_2)$
and $(k_2,k_1)$ to be counted twice).
\begin{align}
\left(    \sigma\right)_\text{cm,nr} = \frac{\alpha^2\pi}{2m
    \left|\textbf{p}\right|}. 
\end{align}
If I were to introduce a Bohr radius $1 / R_b = \alpha m$, I would write this cross section as
\begin{align}
    \boxed{
        \left(\sigma\right)_\text{cm,nr} = \frac{\alpha^4\pi}{2}
        \frac{m}{\left|\textbf{p}\right|} R_b^2.
        }
\end{align}
\subsection{Comparison with experiments}%
The data from LEP involves high energy processes ($E_\text{cm} \sim \SI{100}{\giga\ev}$).
For what follows I used Python, which does not like very large numbers. Because $m \approx
\SI{0.5}{\mega\ev} $ and, on top of that, all the energies in the differential cross section are squared,
it is a valid approximation to consider massless electrons:
\begin{align}
 m^2\ll   \textbf{p}^2 \approx E^2 .
\end{align}
There is a small caveat in that we actually have factors of $\textbf{p}^2 \cos^2\theta$
and $\textbf{p}^2 \sin^2\theta$ which could become arbitrarily small for the right values
of theta, rendering the UR approximation invalid. However the values for $\theta$
considered in the paper are far enough from those critical points ($0,\pi,\ldots$).

In the UR limit, 
\begin{align}
    \left(\frac{d\sigma}{d\Omega}\right)_\text{cm,ur} &= \frac{\alpha}{8 E^2}
    \left\{\frac{E^2(3+\cos^2\theta)}{E^2\sin^2\theta}-
    \left(\frac{E^2\sin^2\theta}{E^2\sin^2\theta}\right)^2\right\}\nonumber \\ 
    &= \frac{\alpha}{8E^2}\frac{3 + \cos^2\theta - 1}{\sin^2\theta}\nonumber \\ 
    &= \frac{\alpha}{8E^2}\left(\frac{1+\cos^2\theta}{1-\cos^2\theta}\right).
\end{align}
To compare with data, I also needed to convert from 1/GeV$^2$ to pb:
\begin{align}
   \SI{1}{\pico\barn}  &= 10^{-12}\left( \SI{100}{\femto\metre\squared}  \right)  =
   \frac{\SI{e-10}{\femto\metre}}{\left( \hbar c \right) ^2} =
   \frac{\SI{e-10}{\femto\metre^2}
   }{\left( \SI{0.197}{\giga\ev\femto\metre} \right) ^2 } 
\end{align}
which can be promptly inverted for 1/GeV$^2$:
\begin{align}
    \frac{1}{\text{GeV}^2} = \SI{0.389e10}{\pico\barn} .
\end{align}
For the total cross section, I integrated with scipy the relativistic expression for the
differential cross section over the given angle range. In principle there is a divergence at $\theta= 0$, but we only
need to calculate the cross section between $\theta \in R = \left[25^{\circ}, 35^{\circ}\right]\cup
\left[43^{\circ},88^{\circ}\right]$:
\begin{align}
    \sigma^{0} &= 2\pi \frac{\alpha^2 }{4E^2} \int_{R} \mathop{d\cos\theta}
    \frac{1+\cos^2\theta}{1-\cos^2\theta}.
\end{align}
The integral is just a numerical factor and we have a dependency $\sigma^{0} \propto
E_\text{cm} ^{-2}$.

I plotted the results of my comparison in figures \ref{fig:qed1}, \ref{fig:qed2}.

\begin{figure}[htpb]
\centering
\includegraphics[width=0.7\linewidth]{qed1.pdf}
\caption{Differential cross section in UR limit vs data.}
\label{fig:qed1}
\end{figure}
\begin{figure}[htpb]
\centering
\includegraphics[width=0.7\linewidth]{qed2.pdf}
\caption{Total cross section in UR limit vs data.}
\label{fig:qed2}
\end{figure}
\newpage
\section{Exercise 5}%
\begin{exercise}
Consider an extension of the QED theory in which, in addition to the fermion
$f$ of mass $m$ and to the massless photons, a pseudo-scalar particle ($\phi$) of mass
$\mu$ exists. The interaction Lagrangian of such a theory would be
\begin{align}
    \mathcal{L}_\text{I}  = -i e \bar{\psi} \gamma^{\mu} \psi A_\mu - ig
    \bar{\psi}\gamma^{5}\psi \phi.
\end{align}
\end{exercise}
\subsection{Feynman rules}%
I build the Feynman rules by analogy with QED and from what I know concerning scalar
fields. First of all, the propagators in momentum space:
\begin{align}
\wick{\c1 \phi(x) \c1\phi(y)} &= 
\begin{gathered}
   \feynmandiagram [horizontal=a to b] {
   a --[scalar, momentum = \(q\)] b
   }; 
\end{gathered} = \frac{i}{q^2 - \mu^2 +i\varepsilon}\nonumber \\ 
\wick{\c1\psi(x) \bar{\c1\psi}(y)} &= 
\begin{gathered}
   \feynmandiagram [horizontal=a to b] {
   a --[fermion, momentum = \(q\)] b
   }; 
\end{gathered} = \frac{i( \slashed{p}+m)}{p^2-m^2+i\varepsilon}\nonumber \\ 
\wick{\c1A_{\mu}(x) \c1A_\nu(y)} &= 
\begin{gathered}
   \feynmandiagram [horizontal=a to b] {
   a --[photon, momentum = \(q\)] b
   }; 
\end{gathered} = -\frac{i}{p^2 + i\varepsilon}g_{\mu\nu} \quad\text{(Feynman gauge)}.
\end{align}
The external legs are trivial:
\begin{align}
    \wick{\c1\phi\vert\c1 p\rangle}  &= 
    \begin{gathered}
        \feynmandiagram [horizontal=a to b] {
        a -- [scalar]b[dot]
        };
    \end{gathered} = 1 & 
 \wick{  \langle \c1 p \vert \c1\phi}  &= 
    \begin{gathered}
        \feynmandiagram [horizontal=a to b] {
        a[dot] -- [scalar]b
        };
    \end{gathered} =1\nonumber \\ 
    \wick{\c1\psi \vert e^{-};\c1 p, s \rangle} &= 
    \begin{gathered}
        \feynmandiagram [horizontal=a to b] {
        a -- [fermion]b[dot]
        };
    \end{gathered} = u^{s}(p) & 
        \wick{ \langle e^{-}; \c1 p, s \vert\bar{\c1\psi}}  &= 
    \begin{gathered}
        \feynmandiagram [horizontal=a to b] {
        a[dot] -- [fermion]b
        };
    \end{gathered} =\bar{u}^{s}(p)\nonumber \\ 
    \wick{\c1{\bar{\psi}} \vert e^{+}; \c1 p, s \rangle} &= 
    \begin{gathered}
        \feynmandiagram [horizontal=a to b] {
        a -- [anti fermion]b[dot]
        };
    \end{gathered} = \bar{v}^{s}(p) & 
        \wick{ \langle e^{+}; \c1 p, s\vert\c1\psi}  &= 
    \begin{gathered}
        \feynmandiagram [horizontal=a to b] {
        a[dot] -- [anti fermion]b
        };
    \end{gathered} =\bar{u}^{s}(p)\nonumber \\ 
    \wick{\c1 A_\mu \vert \c1 p \rangle}  &= 
    \begin{gathered}
        \feynmandiagram [horizontal=a to b] {
        a -- [photon]b[dot]
        };
    \end{gathered} = \varepsilon_\mu (p) & 
    \wick{\langle \c1 p \vert\c1 A_\mu }  &= 
    \begin{gathered}
        \feynmandiagram [horizontal=a to b] {
        a[dot] -- [photon]b
        };
    \end{gathered} = \varepsilon^\star_\mu (p).
\end{align}
For the vertices, we still have the standard QED vertices (2 fermionic lines, 1 photon
line); similarly, since the scalar coupling term has 3 fields, LSZ + Wick's theorem tell
me that there must be three legs for a nonvanishing contribution to the scattering
amplitude.  Then I can just spew out the result. 
\newpage
\begin{align}
\begin{gathered}
    \feynmandiagram [horizontal=a to b] {
    i1 -- [fermion] a -- [fermion] i2, 
    a -- [photon]b [particle = \(\mu\)] 
    };
\end{gathered} &= -ie \gamma^{\mu},&
\begin{gathered}
    \feynmandiagram [horizontal=a to b] {
    i1 -- [fermion] a -- [fermion] i2, 
    a -- [scalar]b     };
\end{gathered} &= - i g \gamma^{5}.
\end{align}
The rest is the usual.
\begin{itemize}
    \item Impose 4-momentum conservation at each vertex ($2\pi^{4}\delta^{(4)}(\ldots)$).
    \item For each free 4-momentum $q$, integrate over $\int
    \frac{\mathop{d^4q}}{(2\pi)^4}$.
    \item For each fermionic loop or exchange of identical fermions, add a minus sign. 
\end{itemize}
\subsection{Feynman diagrams}%
The initial and final states for 
\begin{align}
    \gamma e^{-}\to e^{-} \pi
\end{align}
(where I gave the pseudoscalar particle a name) are
\begin{align}
    \ket{\Phi_\alpha}  &= \ket{\gamma; \lambda,k} \otimes \ket{e^{-}; p,s} = \sqrt{2E_k
    2E_p} a^\dagger (k,\lambda) c^\dagger (p,s) \ket{0} \nonumber \\ 
    \ket{\Phi_\beta}  &= \ket{\pi; k^\prime} \otimes \ket{e^{-}; p^\prime,s^\prime} =
    \sqrt{2E_{k^\prime} 2E_{p^\prime}} \pi^\dagger (k^\prime) c^\dagger
    (p^\prime,s^\prime) \ket{0} .
\end{align}
At leading order, I found two diagrams:
\begin{align}
i \mathcal{M} = 
\underbrace{    \begin{gathered}
        \feynmandiagram [horizontal=a to b] {
        i1 [particle=\(u^{s}(p)\)] -- [fermion] a -- [photon] i2 [particle={\(\varepsilon(
        k,\lambda) \)}],
        a -- [fermion, momentum'=\(q\)] b,
        f1 [particle=\(\pi(k^\prime)\)] -- [scalar] b -- [fermion] f2
        [particle=\(u^{s^\prime}(p^\prime)\)],
        };
    \end{gathered}}_{i\mathcal{M}_s}+
    \underbrace{\begin{gathered}
                    \feynmandiagram [vertical=a to b] {
                    i1 [particle = {\(\varepsilon(k, \lambda)\)}] -- [photon] a ,
                    i2 [particle = \(u^{s}(p)\)] -- [fermion]b -- [fermion,momentum =
                    \(\tilde{q}\)] a,
                    b -- [scalar] f2 [particle = \(\pi (k^\prime)\)] ,
                    a -- [fermion] f1 [particle = \(u^{s^\prime}(p^\prime)\)] , 
                    };
                \end{gathered}}_{i\mathcal{M}_u}.
\end{align}
As a sidenote: I think I used an unconventional form for the Mandelstam variables, as the
channel I called u is more of a t channel. Also, we do not need to include both t and u
channels as the final state particles are distinguishable.

Going upstream as usual,
\begin{align}
    \mathcal{M}_s &= - \frac{eg}{q^2-m^2 +i\varepsilon} \bar{u}^{s^\prime} (p^\prime)
    \gamma^{5} (\slashed{q} + m) \gamma^{\mu} u^{s}(p) \varepsilon_\mu (k,\lambda),\nonumber \\ 
    \mathcal{M}_u &= - \frac{eg}{\tilde{q}^2-m^2 +i\varepsilon} \bar{u}^{s^\prime} (p^\prime)
    \gamma^{\mu} (\tilde{\slashed{q}} + m) \gamma^{5} u^{s}(p) \varepsilon_\mu
    (k,\lambda), \label{5.199}
\end{align}
and 4-momentum conservation dictataes 
\begin{align}
    q &= p + k = p^\prime +k^\prime\nonumber \\ 
    \tilde{q} &= p-k^\prime = p^\prime - k.
\end{align}
\subsection{Unpolarised square matrix element}%
The procedure here reads as in the last exercise. We have to compute 3 Dirac traces.
There is one key difference when taking the h.c. of the scattering amplitude: 
\begin{rmk}
Here the h.c. picks up a minus sign due to gamma 5 algebra.
\end{rmk}
\begin{proof}
Relies on $\gamma^{5}$ anticommuting with $\gamma^{\mu}$ and it being hermitian.
    \begin{align}
        \mathcal{M}^\dagger &\propto \left[\bar{u}^{s^\prime}\gamma^{5}(\slashed{q} + m)
        \gamma^{\mu} u^{s}\right] \nonumber \\ 
        &= \left(u^{s}\right)^\dagger \gamma^{0}\gamma^{\mu}\cancel{\gamma^{0}} (q_\rho
       \cancel{ \gamma^{0}}\gamma^{\rho}\gamma^{0} + m)\gamma^{5} \gamma^{0}u^{s}\nonumber \\ 
        &= \bar{u}^{s}\gamma^{\mu} (\slashed{q}+m)
        \underbrace{\gamma^{0}\gamma^{5}\gamma^{0}}_{-\gamma^{5}} u^{s}\nonumber \\ 
        &= - \bar{u}^{s} \gamma^{\mu} \left(\slashed{q}+m\right) \gamma^{5}u^{s}.
    \end{align}
\end{proof}
\subsubsection{Trace 1/3}%
The square of $\mathcal{M}_s$ is 
\begin{align}
    \left|\mathcal{M}_s\right|^2 &= -\frac{g^2e^2}{(q^2-m^2)^2} \bar{u}^{s^\prime}(p^\prime )
    \gamma^{5}(\slashed{q}+m) \gamma^{\mu}u^{s}(p) \bar{u}^{s}(p) \gamma^{\mu^\prime }
    (\slashed{q}+m ) \gamma^{5}u^{s^\prime }(p^\prime) \varepsilon_\mu (k,\lambda)
    \varepsilon_{\mu^\prime }(k,\lambda ).
\end{align}
Then we ought to sum over spin and polarisation states and normalise over possible inital
states (2 spin states $\times $ 2 helicity states). At the end, I get a trace; spinor completeness gives me
a $(\slashed{p}^\prime + m) $ and a $(\slashed{p}+m)$, polarisation sums give me a
$g_{\mu\mu^\prime }$ and the averaged amplitude is
\begin{align}
    \left|\overline{\mathcal{M}}_u\right|^2 &= -\frac{g^2 e^2}{4(q^2-m^2)^2}
    \Tr\left\{\mathbin{\color{teal}(\slashed{p}^\prime + m ) \gamma^{5}}
    (\slashed{q}+m)\mathbin{\color{magenta}
    \gamma^{\mu}(\slashed{p}+m) \gamma_\mu} (\slashed{q}+m)\mathbin{\color{teal}\gamma^{5}}\right\}.
\end{align}
Workable. I move the last $\mathbin{\color{teal}\gamma^{5}}$ over to the left and terms in
teal become
\begin{align}
\mathbin{\color{teal}    \gamma^{5} (\slashed{p}^\prime+m)\gamma^{5}}&\mathbin{\color{teal}= \gamma^{5}\gamma_\rho
    \gamma^{5}(p^\prime)^{\rho} + m (\gamma^{5})^2}\nonumber \\ 
    & \mathbin{ \color{teal}=-\gamma_\rho (p^\prime)^{\rho} + m= (-\slashed{p}^\prime+m)
}\end{align}
(thanks to $\gamma^{5}\gamma^{5}= 1$ and $\{\gamma^{5},\gamma^{\mu}\} = 0$). The terms in
magenta become (according to \eqref{gronz})
\begin{align}
    \mathbin{\color{magenta}\gamma^{\mu}(\slashed{p}+m)\gamma_\mu = -2\slashed{p} + 4m}
\end{align}
and our trace becomes
\begin{align}
\left|\overline{\mathcal{M}}_s\right|^2  =-\frac{e^2 g^2}{4 (q^2-m^2)^2}2 \Tr \left\{(\slashed{p}^\prime - m) (\slashed{q}+m)
    (\slashed{p} - 2m) (\slashed{q} +m)\right\}.
\end{align}
Forgetting temporarily about the prefactor, I decompose the trace (keeping only even
numbers of gamma matrices):
\begin{align}
\Tr\left\{1\right\}  &= \Tr\left\{\slashed{p}^\prime \slashed{q} \slashed{p}
\slashed{q}\right\}+ \Tr \left\{\slashed{p}^\prime \slashed{q} (-2m) m\right\} +
\Tr\left\{\slashed{p}^\prime m \slashed{p} m\right\} +\Tr\left\{\slashed{p}^\prime m (-2m)
\slashed{q}\right\}+ \nonumber \\ 
&+ \Tr\left\{(-m) \slashed{q} \slashed{p} m\right\} + \Tr\left\{(-m) \slashed{q} (-2m)
\slashed{q}\right\} + \Tr\left\{(-m)m
\slashed{p}\slashed{q}\right\}+\Tr\left\{(-m)m(-2m)m\right\}\nonumber \\ 
&= \Tr\left\{\slashed{p}^\prime \slashed{q}\slashed{p}\slashed{q}\right\} + m^2 \Tr\left\{\slashed{p}^\prime
\slashed{p}\right\} - 4m^2 \Tr\left\{\slashed{p}^\prime \slashed{q}\right\} + 2m^2
\Tr\left\{\slashed{q}\slashed{q}\right\} -2m^2 \Tr\left\{\slashed{p}\slashed{q}\right\} +
2m^{4}\Tr \mathbb{I} .
\end{align}
The only term that needs some work is the first one. Using 
\begin{align}
\Tr \gamma^{\mu}\gamma^{\nu} \gamma^{\alpha}\gamma^{\beta} =
4\left(g^{\mu\nu}g^{\alpha\beta} - g^{\mu\alpha}g^{\nu\beta} +
g^{\mu\beta}g^{\nu\alpha}\right),
\end{align}
substituting $g^{\mu\nu} \to  \cdot $,
\begin{align}
    \Tr \slashed{p}^\prime \slashed{q} \slashed{p} \slashed{q} &= 4 \left[\left(p^\prime
    \cdot q\right)\left(p \cdot  q\right) - \left(p^\prime\cdot p\right)\left(q\cdot
    q\right) + \left(p^\prime\cdot q\right)\left(q\cdot p\right)\right]\nonumber \\ 
    &= 4 \left[2 (p^\prime q) (pq) - q^2(pp^\prime ) \right].
\end{align}
Thus
\begin{align}
    \Tr\left\{1\right\}&= 4 \left[2(p^\prime q) (pq ) - q^2(p^\prime p )
    + m^2 ( p^\prime p - 4 p^\prime q + 2 q^2 - 2 pq) + 2m^{4}\right] \nonumber \\ 
    &= 4 \left\{ 2 [p^\prime \cdot  (p^\prime+ k^\prime)][p\cdot  (p+k)] - q^2 pp^\prime +
    m^2 [-4 p^\prime(p^\prime+k^\prime) - 2p (p+k) +p p^\prime +2 q^2 ] +2m^{4}\right\}\nonumber \\ 
    &= 4 [ 2 (m^2 + p^\prime k^\prime) (m^2 + pk) - q^2 pp^\prime + m^2 ( -4m^2 -
    4p^\prime k^\prime -2m^2 - 2pk + pp^\prime +2q^2 ) +2m^{4}]\nonumber \\ 
&= 4 \{ 2m^4 + 2m^2 (\cancel{pk}+ p^\prime k^\prime) + 2(pk)(p^\prime k^\prime) - q^2 pp^\prime
+m^2 (-4p^\prime k^\prime -\cancel{ 2pk} + pp^\prime +2q^2 ) -4m^{4} \}\nonumber \\ 
&= 4 [2 (p^\prime k^\prime ) (p k) - q^2 p p^\prime + m^2
p p^\prime - 2m^2 p^\prime k^\prime +2m^2q^2-2m^4 ].
\end{align}
In terms of the Mandelstam:
\begin{align}
s & = (p+k)^2= q^2 & \implies  q^2 &= s\nonumber \\ 
    s &= (p+k)^2 = m^2 + pk & \implies pk &= \frac{1}{2}(s-m^2)\nonumber \\ 
&= (p^\prime+k^\prime)^2 = m^2 +\mu ^2 + 2p^\prime k^\prime & \implies p^\prime k^\prime
&= \frac{1}{2}(s-m^2-\mu^2) \nonumber \\ 
t &= (p-p^\prime) ^2 = 2m^2 - 2pp^\prime & \implies pp^\prime &= \frac{1}{2}(2m^2 - t)\nonumber \\ 
&&s+u+t &= 2m^2 +\mu^2
\end{align}
and the trace becomes
\begin{align}
\Tr\left\{1\right\} &= 4(2(pk-m^2)p^\prime k^\prime + (m^2-q^2)p p^\prime + 2m^2q^2
-2m^{4}) \nonumber \\ 
&= 4 [ 2 (\frac{s-m^2}{2}-m^2)\frac{1}{2}(s-m^2-\mu^2) + \frac{1}{2}(m^2-s)(2m^2-t) +
2m^2s - 2m^{4})\nonumber \\ 
&= 4[ \frac{1}{2}(s-3m^2)(s-m^2-\mu^2) + m^{4} - \frac{1}{2}m^2t - m^2s +\frac{1}{2}st
+2m^2s - 2m^{4}]\nonumber \\ 
&= 4[ \frac{1}{2}(s^2-m^2s -\mu^2s -3m^2s +3m^{4}+3m^2\mu^2) - m^{4} - \frac{1}{2}m^2t
+m^2s + \frac{1}{2}s t]\nonumber \\ 
&= 2(s^2- 4m^2s - \mu^2s +3m^{4} +3m^2\mu^2) - 4m^{4} -2m^2t + 4m^2s +2st\nonumber \\ 
&= 2s^2 -4m^2s - 2\mu^2 s - 2m^2 t +2st +2m^{4}+6m^2\mu^2.
\end{align}
This looks nicer if I remove the $t = 2m^2 + \mu^2 - u - s$: we have:
\begin{align}
    -2m^2t &= -2m^2 (2m^2 + \mu^2- s-u) = -4m^{4}-2m^2\mu^2 +2m^2 - s +2m^2u\nonumber \\ 
2st&= 2s (2m^2 +\mu^2 - s-u) = 4m^2s + 2\mu^2 s -2s^2 - 2su
\end{align}
and I get
\begin{align}
    \boxed{
        \Tr\left\{1\right\} = 4 (m^2(s + u)  -su+2m^2\mu^2- m^{4} )
        .}
\end{align}

\subsubsection{Trace 2/3}%
The second trace would be the square of the $u$ channel: 
\begin{align}
    \left|\overline{\mathcal{M}}_u\right|^2 &= \frac{e^2g^2}{4(\tilde{q}^2-m^2)^2}
    \bar{u}^{s^\prime }(p^\prime) \gamma^{\mu} (\slashed{q} + m ) \gamma^{5}u^{s}(p)
    (-\gamma^{5}) (\slashed{q}+m)\gamma_\mu \bar{u}^{s^\prime}(p^\prime)\nonumber \\ 
    &= -\frac{e^2g^2}{4(\tilde{q}^2-m^2)^2} \Tr\left\{(\slashed{p}^\prime + m)
    \gamma^{\mu}(\tilde{\slashed{q}}+m) \gamma^{5}(\slashed{p}+m)\gamma^{5}
    (\tilde{\slashed{q}}+m)\gamma_\mu\right\}.
\end{align}
The gamma matrix order in the previous trace was $\gamma^{5}\ldots \gamma^{5}\ldots
\gamma^{\mu}\ldots\gamma_\mu$; here it is $\gamma^{\mu}\ldots\gamma_\mu \ldots
\gamma^{5}\ldots\gamma^{5}$ ($ \gamma^{\mu}\gamma_\mu = \gamma_\mu \gamma^{\mu}$). So here I make a cyclic permutation:
\begin{align}
    \left|\overline{\mathcal{M}}_u\right|^2 &= -\frac{e^2g^2}{4 (\tilde{q}^2 - m^2)^2}
    \Tr\left\{\gamma^{5} (\slashed{p} + m) \gamma^{5} (\tilde{\slashed{q}}+m)\gamma^{\mu}
    (\slashed{p}^\prime +m) \gamma_\mu (\tilde{\slashed{q}}+m)\right\}.
\end{align}
Now I can paste the previous result with the formal substitutions 
\begin{align}
    p &\leftrightarrow p^\prime ,\nonumber \\ 
    q & \leftrightarrow \tilde{q}.
\end{align}
The second line tells me I have to substitute $\tilde{q}= p-k^\prime $ in place for $q =
p^\prime + k^\prime \to  p + k^\prime$ which is satisfied if I substitute $k^\prime\to  -
k^\prime$, and likewise: $\tilde{q} = p^\prime - k $ in place of $q = p + k \to
p^\prime +k $ means $k\to -k$. In terms of the Mandelstams, I get
\begin{align}
    s &= (p+k)^2 \to  (p^\prime-k)^2 = u \nonumber \\ 
    t &= (p-p^\prime)^2 \to  (p^\prime-p)^2 = t \nonumber \\ 
    u &= (p-k^\prime ) ^2 \to  (p^\prime + k^\prime) = s
\end{align}
which perchance is excessive yapping to say that the $u$ channel is obtained by switching
$s\leftrightarrow u$. Since the trace is symmetric wrt these two variables, I get the same
result:
\begin{align}
    \boxed{
        \Tr\left\{2\right\} = 4 (m^2(s +u)  -su+2m^2\mu^2- m^{4} ).
        }
\end{align}

\subsubsection{Trace 3/3}%
The interference term. I read it off \eqref{5.199}.
\begin{align}
  \frac{1}{4}  \mathcal{M}_s \mathcal{M}_u^\star &= \frac{e^2g^2}{4(s-m^2)(u-m^2)} (-)\Tr
    \left\{(\slashed{p}^\prime +m)\gamma^{5}(\slashed{q} +m) \gamma^{\mu}(\slashed{p}+m )
    \gamma^{5}(\tilde{\slashed{q}}+m) \gamma_\mu\right\}.
\end{align}
Moving the $\gamma^{5}$ around:
\begin{align}
   - \Tr\left\{3\right\} &= -\Tr\left\{\gamma_\mu
    (\slashed{p}^\prime+m)\gamma^{5}(\slashed{q}+m)\gamma^{\mu}(\slashed{p}+m)\gamma^{5}(\tilde{\slashed{q}}+m)\right\}\nonumber \\ 
    &= -\Tr \left\{ \gamma_\mu (\slashed{p}^\prime+ m) (-\slashed{q}
    +m)\gamma^{5}\gamma^{\mu}\gamma^{5}(-\slashed{p}+m) (\tilde{\slashed{q}}+m)\right\}\nonumber \\ 
    &= +\Tr\{\underbrace{\gamma_\mu (\slashed{p}^\prime+m)
    (\slashed{q}-m)\gamma^{\mu}}_{A}(\slashed{p}-m) (\tilde{\slashed{q}}+m)\}
\end{align}
so this time around I can forget about the minus sign. As in the previous exercise, $A$
becomes
\begin{align}
    A &= -m^2 \gamma_\mu \gamma^{\mu} + m \gamma_\mu (-\slashed{p}^\prime +
    \slashed{q})\gamma^{\mu} + \gamma_\mu \slashed{p}^\prime \slashed{q}\gamma^{\mu} .
\end{align}
Plugging the usual Dirac algebra \eqref{dalgebra}, 
\begin{align}
    A &= -4m^2 +2 m (\slashed{p}^\prime-\slashed{q}) +4p^\prime\cdot q
\end{align}
and the trace is (keeping only even numbers of gamma matrices in the second line)
\begin{align}
   \Tr\left\{3\right\}  &= \Tr\left\{[4(p^\prime q - m^2) +
   2m(\slashed{p}^\prime-\slashed{q}]x\left[(\slashed{p}-m)(\tilde{\slashed{q}}+m)\right]
   \right\}\nonumber \\ 
   &= 4(-m^2 + p^\prime q) \Tr\left(\slashed{p} \tilde{\slashed{q}}-m^2\right) + 2m^2
   \Tr\left\{(\slashed{p}^\prime-q)(\slashed{p}-\tilde{\slashed{q}})\right\}\nonumber \\ 
   &= 16(-m^2 +p^\prime q)(p \tilde{q} - m^2) +2m^2 (p^\prime-q)(p- \tilde{q})\nonumber \\ 
   &= 16 (p^\prime q)(p \tilde{q}) + 16m^{4} + 8m^2
   ( -2p\tilde{q} -  2p^\prime q +p^\prime p - p^\prime \tilde{q} - qp +q\tilde{q}). 
\end{align}
As usual, $q = k+p = k^\prime +p^\prime $, $\tilde{q} = p-k^\prime = p^\prime-k$. In terms
of the Mandelstam,
\begin{align}
    p^\prime q &= p^\prime (p^\prime+k^\prime) = m^2 +p^\prime k^\prime = m^2 +
    \frac{1}{2}(s-\mu^2 -m^2)
    \nonumber \\ 
    &= \frac{1}{2}(m^2 - \mu^2 +u) \nonumber \\ 
    qp &= p(p+k) = m^2 + pk = m^2 + \frac{1}{2}(s-m^2) \nonumber \\ 
    &= \frac{1}{2}(s+m^2)\nonumber \\ 
    p^\prime p &= \frac{1}{2}(2m^2 - t) 
\end{align}
and substituting $p \leftrightarrow p^\prime, $ $q \leftrightarrow \tilde{q}$ on the LHS, I get
$u\leftrightarrow s$ on the RHS so these follow painlessly:
\begin{align}
    p \tilde{q} &= \frac{1}{2}(u - \mu^2 +m^2)\nonumber \\ 
    \tilde{q} p^\prime &= \frac{1}{2}(u+m^2)
\end{align}
and finally
\begin{align}
    q\tilde{q}&= (p-k^\prime)(p+k) = m^2- k^\prime p +pk - kk^\prime\nonumber \\ 
    &= m^2 - \frac{1}{2}(m^2 +\mu^2 - u + s - m^2 - \mu^2 - t)\nonumber \\ 
    &= \frac{1}{2}(-2\mu^2 + u + t + s).
\end{align}
Performing all due substitutions, 
\begin{align}
    \Tr{3} &= 4 (s-\mu^2 + m^2)(m^2-\mu^2+u)+ 16m^{4}+ 4m^2 (-4m^2 + 4\mu^2 -2 u -2 s + \nonumber \\ 
    & \phantom{=(} + 2m^2 - \cancel{t} - s - m^2 - u - m^2 - 2 \mu^2 +u + s
    +\cancel{t})\nonumber \\ 
    &=  4 \left(m^2 s - \mu^2s + us - \cancel{\mu^2m^2 }+ \mu^{4}- \mu^2 u + m^{4} - \cancel{m^2 \mu^2} +
    m^2u\right)  + 16m^{4} +   \nonumber \\  
    & \phantom{=} + 4m^2 ( -4m^2 +\cancel{2\mu^2 }  -2 u -2s ) \nonumber \\ 
    &= 4 (- m^2 s - \mu^2 s -m^2 u - \mu^2 u + us  + m^{4}+\mu^{4})
\end{align}
or 
\begin{align}
    \boxed{
        \Tr\left\{3\right\} = 4 ( - (m^2 + \mu^2)(u+s) + us +m^{4}+\mu^{4}).
        }
\end{align}
\subsubsection{Unpolarised square matrix element}%
Recollecting:
\begin{align}
    \left|\overline{\mathcal{M}}\right|^2 &= \left|\overline{\mathcal{M}}_s\right|^2 +
    \left|\overline{\mathcal{M}}_u\right|^2 + 2 \mathfrak{Re}
    \overline{\mathcal{M}_u^\star\mathcal{M}_s} \nonumber \\ 
    &= -\frac{g^2e^2}{4 (s-m^2)^2} \Tr\left\{1\right\} - \frac{g^2
    e^2}{4(s-m^2)^2}\Tr\left\{2\right\} + 2 \frac{e^2 g^2}{4(s-m^2)(u-m^2)}
    \Tr\left\{3\right\}\nonumber \\ 
    &= (m^{4} - m^2 (u + s) +su - 2m^2\mu^2)\left(\frac{1}{(s-m^2)^2}+
    \frac{1}{(u-m^2)^2}\right) + \nonumber \\ &+ 2 \frac{ - (m^2+\mu^2)(s+u)+su+\mu^{4}+
    m^{4}}{(s-m^2)(u-m^2)}.
\end{align}
I couldn't bring this to a much nicer form. Best I could do is
\begin{align}
    \boxed{
        \left|\overline{\mathcal{M}}\right|^2= (m^{4}- 2m^2 \mu^2 -  m^2(u+s)
        +su)\left(\frac{1}{s-m^2} + \frac{1}{u-m^2}\right)^2 + 2 \frac{-\mu^2 (s+u) +su
        +\mu^{4}+ 2m^2\mu^2}{(s-m^2)(u-m^2)}
        } \label{matrixElement}
\end{align}
which fits in one line (almost). This was a long calculation for which I found no
references, so to save time in case it is wrong: the expressions $\Tr\left\{\ldots\right\}$ I also evaluated with
the FeynCalc package for Mathematica, and their results are correct. Possible mistakes must be
before or after that step.
\subsection{Cross section in lab frame}%
\subsubsection{Cross section expression}%
The $2\to  2$, $pk\to p^\prime k^\prime$ differential cross section is given by 
\begin{align}
   \mathop{d\sigma} &= \frac{1}{2E_p 2E_k \left|v_p - v_k\right|}
   \frac{\mathop{d^3p^\prime}}{(2\pi)^3} \frac{\mathop{d^3k^\prime }}{(2\pi)^3}
   \frac{1}{2E_{p^\prime }} \frac{1}{2E_{k^\prime }}
   \left|\overline{\mathcal{M}}\right|^2 (2\pi)^{4} \delta^{(4)}(p+k-p^\prime-k^\prime ),
\end{align}
where the velocities are defined as $\textbf{v}_p =\textbf{p} / E_p $ etc. In the rest frame of  $p$, $v_p =
0$, kinematics is straightforward:
\begin{align}
  p &= (m,0,0,0)  \nonumber \\ 
  k &= (\left|k\right|\eqqcolon \omega, \textbf{k}) , \quad \left|\textbf{v}_k\right| = 1.
\end{align}
Now I integrate the delta in $\mathop{d^3k^\prime } /(2\pi)^3 $:
\begin{align}
   \mathop{d\sigma} &= \frac{1}{2m 2 \omega}  \frac{\mathop{d^3p^\prime }}{(2\pi)^3}
   \frac{1}{2E_{p^\prime }} \frac{1}{2E_{k^\prime }}
   \left|\overline{\mathcal{M}}\right|^2 (2\pi) \delta (m + \omega - E_{p^\prime } -
   E_{k^\prime }) \nonumber \\ 
  \frac{d\sigma}{d\Omega} &=\frac{1}{4m\omega} \int \frac{\mathop{dp^\prime } }{(2\pi)^2}
  \frac{1}{2E_{p^\prime }2E_{k^\prime }}
  (p^\prime )^2
  \delta( E_{p^\prime  } + E_{k^\prime } - m - \omega)
  \left|\overline{\mathcal{M}}\right|^2 .
\end{align}
Now I employ the usual $\delta$ identity, 
\begin{align}
    \delta (E_{p^\prime } + E_{k^\prime  } - m - \omega) = \delta ( ( (\textbf{p}^\prime
    )^2+ m^2)^{1
    /2}  +( (\textbf{k}^\prime )^2 +\mu^2)^{1 /2} - m - \omega)
\end{align}
(where $\textbf{k}^\prime  + \textbf{p}^\prime  = \textbf{k}$ from the spatial delta):
\begin{align}
    \delta (f(p^\prime)) &= \delta ( ( (\textbf{p}^\prime
    )^2+ m^2)^{1
    /2}  +( (\textbf{k}- \textbf{p}^\prime  )^2 +\mu^2)^{1 /2} - m - \omega)\nonumber \\ 
    &= \frac{1}{\left|f^\prime (p_f)\right|} \delta(p^\prime - p_f), 
\end{align}
where $p_f$ is the momentum that satisfies energy conservation ($f(p_f) = 0$) and 
\begin{align}
    f^\prime &= \frac{d}{dp^\prime }\left(\sqrt{(\textbf{p}^\prime) ^2 +m^2}  +
    \sqrt{(\textbf{p}^\prime )^2 - 2p^\prime
    k\cos\theta + k^2 +\mu^2} - E_k - \mu \right)\nonumber \\ 
    &= \frac{p_f}{E_{p_f}} + \frac{p_f - k \cos\theta }{E_{k^\prime }}.
\end{align}
All together:
\begin{align}
    \cfrac{d\sigma}{d\Omega} = \cfrac{1}{4m\omega} \cfrac{p_f^2}{(2\pi)^2} \cfrac{1}{p_f}
    \frac{1}{2E_{p_f} 2E_{k^\prime }}
    \cfrac{1}{\cfrac{1}{E_{p_f}}+ \cfrac{1-\frac{\omega}{p_f}\cos\theta}{E_{k^\prime }}}
    \left|\overline{\mathcal{M}}\right|^2 
\end{align}
or better
\begin{align}
    \boxed{
        \frac{d\sigma}{d\Omega} =  \frac{1}{64\pi^2 m} \frac{p_f}{\omega}
        \left(E_{k^\prime }+ E_{p_f}(1- \frac{\omega}{p_f}\cos\theta)\right)^{-1}
        \left|\overline{\mathcal{M}}\right|^2 
        } \label{crossSec}
\end{align}
with 
\begin{align}
    E_{k^\prime } &= \left( (\textbf{k}-\textbf{p}_f )^2 + \mu^2\right)^{1 /2}\nonumber \\ 
&= (\omega^2 + p_f^2 -2 \omega p_f \cos\theta + \mu^2)^{1 /2}
\end{align}
\subsubsection{Kinematics}%
(From now on I use $p_f$ instead of $p^\prime $ for the fermionic momentum after
scattering.) The Mandelstam variables in our reference frame become
\begin{align}
    s &= (p+k)^2 = m^2 + 2pk = m^2 + 2m\omega\nonumber \\ 
    u &= (p -k^\prime )^2 = m^2 + \mu^2 - 2p\cdot  k^\prime  = m^2 + \mu^2 - 2mE_{k^\prime
    }.\label{242}
\end{align}
In the cross section we also have some $p_f, E_{p_f}$ factors. These can be traced back to
$\omega, \cos\theta$ from energy conservation:
\begin{align}
    m + \omega &= E_{p_f} + E_{k^\prime }\nonumber \\ 
    &= \left(p_f^2 + m^2 \right)^{1 /2} + \left(\omega^2 + p_f^2 - 2 \omega p_f \cos\theta
    + \mu^2\right)^{ 1/2}.
\end{align}
This is not very attractive, however we can move everything to the LHS and get an
expression of the type $f(p_f, \omega, \cos\theta)= 0$ which I inverted using Mathematica.
Unfortunately, we have a quadratic equation and thus two solutions:
\begin{align}
p_{f,\pm } &= \frac{\omega \left(2 m^2 - \mu^2 + 2 m \omega\right) \cos(\theta) \pm \sqrt{(m +
\omega)^2 \left[\mu^2 (-4 m^2 + \mu^2 - 4 m \omega) + 4 m^2 \omega^2
\cos^2(\theta)\right]}}{2 (m + \omega)^2 - 2 \omega^2 \cos^2(\theta)}.\label{243}
\end{align}
Moreover we have to enforce $p_f \ge 0 $ which yields a jungle of possible intervals
depending on $m,\mu,\omega$. For example, the first solution is fine for 
\begin{itemize}
    \item For \( m > 0 \) and \( 2 m < \mu \):
    \begin{align*}
        & \left( \omega > 0 \quad \land \quad 4 (m + \omega) < \frac{\mu^2}{m} \quad \land \quad -1 \leq \cos(\theta) \leq 1 \right) \\
        & \lor \left( m + \omega = \frac{\mu^2}{4 m} \quad \land \quad -1 \leq
        \cos(\theta) < 0 \right) \\
        & \lor \left( 4 (m + \omega) > \frac{\mu^2}{m} \quad \land \quad \frac{\mu^2}{m} > 2 (\mu + \omega) \quad \land \quad -1 \leq \cos(\theta) \leq -\frac{1}{2} \sqrt{\frac{\mu^2 (4 m^2 - \mu^2 + 4 m \omega)}{m^2 \omega^2}} \right) \\
        & \lor \left( 2 (\mu + \omega) = \frac{\mu^2}{m} \quad \land \quad \cos(\theta) = -1 \right) \\
        & \lor \left( \omega = \mu + \frac{\mu^2}{2 m} \quad \land \quad \cos(\theta) = 1 \right) \\
        & \lor \left( \omega > \mu + \frac{\mu^2}{2 m} \quad \land \quad \frac{1}{2} \sqrt{\frac{\mu^2 (4 m^2 - \mu^2 + 4 m \omega)}{m^2 \omega^2}} \leq \cos(\theta) \leq 1 \right)
    \end{align*}

    \item For \( 2 m \geq \mu \):
    \begin{align*}
        & \left( \omega = \mu + \frac{\mu^2}{2 m} \quad \land \quad \cos(\theta) = 1 \right) \\
        & \lor \left( \omega > \mu + \frac{\mu^2}{2 m} \quad \land \quad \frac{1}{2}
        \sqrt{\frac{\mu^2 (4 m^2 - \mu^2 + 4 m \omega)}{m^2 \omega^2}} \leq \cos(\theta)
        \leq 1 \right).
    \end{align*}
\end{itemize}
Something similar happens for the second solution.
\begin{itemize}
    \item For \( \mu > 0 \):
    \begin{align*}
        & \left( \omega = \mu + \frac{\mu^2}{2 m} \quad \land \quad \cos(\theta) = 1 \quad \land \quad (2 m \geq \mu \lor m > 0) \right) \\
        & \lor \left( m > 0 \quad \land \quad \left( 2 m < \mu \quad \land \quad \left( \left( 2 (\mu + \omega) = \frac{\mu^2}{m} \quad \land \quad 1 + \cos(\theta) = 0 \right) \quad \lor \right. \right. \right. \\
        & \left. \left. \left( \frac{\mu^2}{m} > 2 (\mu + \omega) \quad \land \quad 4 (m + \omega) > \frac{\mu^2}{m} \quad \land \quad 1 + \cos(\theta) \geq 0 \quad \land \quad \sqrt{\frac{\mu^2 (4 m^2 - \mu^2 + 4 m \omega)}{m^2 \omega^2}} + 2 \cos(\theta) \leq 0 \right) \right) \right) \\
        & \lor \left( \omega > \mu + \frac{\mu^2}{2 m} \quad \land \quad \sqrt{\frac{\mu^2 (4 m^2 - \mu^2 + 4 m \omega)}{m^2 \omega^2}} \leq 2 \cos(\theta) \quad \land \quad \cos(\theta) \leq 1 \right)
    \end{align*}
    
    \item For \( 2 m \geq \mu \) and \( \omega > \mu + \mu^2 / (2 m) \):
    \begin{align*}
        & \sqrt{\frac{\mu^2 (4 m^2 - \mu^2 + 4 m \omega)}{m^2 \omega^2}} \leq 2
        \cos(\theta) \quad \land \quad \cos(\theta) \leq 1.
    \end{align*}
\end{itemize}
The bottom line of this is that I shall relinquish all hope of a general, analytical
solution. I move onto Python and work in the very first interval, $2m < \mu
\land 4(m+\omega) < \mu^2 / m$, for which all angles work for $p_{f+}$ and no solution is
valid for $p_{f-}$.  

Knowing $p_f (\omega,\cos\theta)$ I can easily retrieve $E_{p_f} = \left(m^2 +
p_f^2\right)^{ 1/2}$. Now all kinematics is parametrised with $\omega$ and
$\cos\theta$.

I thought it would be nice to plot the differential cross section, since I had to choose
my parameters anyway. I pick $m = \mu / 3$; the second constraint reduces to $\omega < 5 m
/4$. As I'd have hoped, the cross section gets small as $\cos\theta \to 1$,
which corresponds to the fermion being returned to sender, and it is peaked close to
$\cos\theta \sim -1$, which corresponds to $\theta = \pi$.
\begin{figure}[htpb]
\centering
\includegraphics[width=0.7\linewidth]{mod_qed.pdf}
\caption{Differential cross section as function of incoming photon energy.}
\label{fig:mod_qed}
\end{figure}
\subsubsection{Total cross section}%
The total cross section is, in principle,
\begin{align}
    \sigma = 2\pi \int \mathop{d\cos\theta} \frac{d\sigma}{d\Omega}.
\end{align}
With our convoluted expression, an analytical solution would serve no purpose. For the
sake of doing something, I used the data from \ref{fig:mod_qed} and NumPy to find total cross
sections for the hand-picked $m = \mu / 3$ case. 
\begin{table}[htpb]
\centering
\caption{Total cross sections for $m = \mu /3 $}
\label{tab:tot_cross_sec_mod}
\begin{tabular}{|c|c|}
 $\omega / m$ & $\sigma m^2$ \\
 \hline
 $0.1$ &  $73.8$ \\
 $0.3$ &  $2.51$ \\
 $0.5$ &  $0.507$ \\
 $0.8$ &  $0.109$ \\
 $1.0$ &  $0.0492$ \\
\end{tabular}
\end{table}

\subsection{Nonrelativistic limit}%
For the NR limit I take a low-energy photon, $\omega \to 0$. Crucially, the momentum in
\eqref{243} does not go to zero but to (I used Mathematica for the limit)
\begin{align}
    p_f \to  \frac{\sqrt{m^2\mu^2 (\mu^2-4m^2)} }{2m^2} + \tilde{\lambda} (\cos\theta) \omega
\end{align}
for some not-very-nice-looking function of the angle $\tilde{\lambda}(\cos\theta)$, whose
expression we shall miraculously not need. Similarly, the energy is
\begin{align}
    E_{p_f} &= \frac{\mu^2 - 2m^2 }{2m} + \lambda (\cos\theta)\omega.
\end{align}
From energy conservation I can find $E_{k^\prime }$:
\begin{align}
    E_{k^\prime }&= m + \omega - E_{k}\nonumber \\ 
    &= 2m + \omega - \frac{\mu^2}{2} - \lambda (\cos\theta)\omega
\end{align}
and the Mandelstams reduce to (from \eqref{242}):
\begin{align}
   s &= m^2 + 2 m\omega\nonumber \\ 
  u &= m^2 + \mu^2 - 2m E_{k^\prime } = 2\mu^2 - 3m^2 + 2m(\lambda (\cos\theta) -
  1)\omega.
\end{align}
The squared matrix element \eqref{matrixElement} is
\begin{align}
 \left|\overline{\mathcal{M}}\right|^2= \underbrace{(m^{4}- 2m^2 \mu^2 -  m^2(u+s)
        +su)}_{(1)}\left(\frac{1}{s-m^2} + \frac{1}{u-m^2}\right)^2 + 2 \underbrace{\frac{-\mu^2 (s+u) +su
        +\mu^{4}+ 2m^2\mu^2}{(s-m^2)(u-m^2)}}_{(2)/(3)}.
\end{align}
The term $(1)$ becomes
 \begin{align}
    (1) &\overset{\mathcal{O}(1)}{=} m^{4}- 2m^2\mu^2 - m^2(m^2 + 2\mu^2 - 3m^2) + m^2
    (2\mu^2-3m^2) \nonumber \\ 
    &= -2m^2\mu^2 + 3m^{4} + 2m^2 \mu^2 - 3m^{4} = 0
\end{align}
so we need to go to higher order:
\begin{align}
    (1) &\overset{\mathcal{O}(\omega)}{=} -m^2 (2m \omega - 2m \omega + \cancel{2m \lambda \omega})
    + m^2 (-2m\omega +\cancel{ 2m \lambda \omega}) + (2\mu^2- 3m^2)(2m\omega) \nonumber \\ 
    &= -2m^3\omega - 6m^3\omega + 4\mu^2 m \omega \nonumber \\ 
    &= (-8m^3 + 4\mu^2 m) \omega.
\end{align}
The second term becomes
\begin{align}
     (2) &\overset{\mathcal{O}(\omega)}{=} -\mu^2 (m^2 + 2\mu^2 - 3m^2) + m^2 (2\mu^2 -
     3m^2 ) + \mu^{4} + 2m^2\mu^2\nonumber \\ 
     &= 6\mu^2 m^2- \mu^{4} - 3m^{4}.
\end{align}
The denominators are of the type
\begin{align}
    \frac{1}{s-m^2} = \frac{1}{2m\omega}
\end{align}
and
\begin{align}
    \frac{1}{u-m^2} = \frac{1}{2(\mu^2 - 2m^2)}.
\end{align}
To leading order in $\omega$:
\begin{align}
    \left|\overline{\mathcal{M}}\right|^2 &= \frac{(-8m^3 + 4\mu^2 m )
    \omega}{4m^2\omega^2} + 2 \frac{6\mu^2 m^2 - \mu^{4} - 3m^{4}}{2m \omega (2\mu^2 -
    4m^{4})}\nonumber \\ 
    &= \left(\frac{-8m^3 + 4\mu^2 m }{4m^2} + \frac{6\mu^2m^2 - \mu^{4} - 3m^{4}}{2\mu^2 -
    4m^{4}}\right) \frac{1}{\omega}.
\end{align}
As usual this is not particularly attractive, however the takeaway is that
$\left|\overline{\mathcal{M}}\right|^2 \sim  1 / \omega$. As promised, the ugliest bits
containing $\lambda(\cos\theta)$ cancel out with I found interesting. The cross section is
\eqref{crossSec}:
\begin{align}
\frac{d\sigma}{d\Omega}=   \frac{1}{64\pi^2m } \frac{p_f}{\omega} \left(E_{k^\prime } + E_{p_f}
   (1-\frac{\omega}{p_f}\cos\theta)\right)^{-1} \left|\overline{\mathcal{M}}\right|^2 .
\end{align}
The factor in parentheses reduces to
\begin{align}
    \left(E_{k^\prime } + E_{p_f}\right)^{-1}  = (m+\omega)^{-1} \to \frac{1}{m}
\end{align}
(since $p_f$ is finite as $\omega\to 0$). Overall, 
\begin{align}
    \boxed{
        \frac{d\sigma}{d\Omega} \overset{\text{NR}}{=} \frac{1}{64\pi^2}\frac{\sqrt{\mu^2
        m^2 (\mu^2-4m^2)} }{2m^4} \left(\frac{\mu^2m - 2m^3}{m^2}+\frac{6\mu^2m^2 -
        \mu^{4}-3m^{4}}{2\mu^2-4m^2}\right) \frac{1}{\omega^2}
        }
\end{align}
and, most notably,
\begin{align}
    \boxed{
        \frac{d\sigma}{d\Omega} \overset{\text{NR}}{\sim}  \frac{1}{\omega^2}.
        }
\end{align}
As foreshadowed in the plot \ref{fig:mod_qed} and in table \ref{tab:tot_cross_sec_mod}, the cross-section blows up in the NR limit.
The total cross section is just
\begin{align}
    \sigma_\text{nr}  = 4\pi \frac{d\sigma_\text{nr} }{d\Omega}
\end{align}
since all angular dependence has vanished. I sincerely hope I didn't make a mistake early on.






\end{document}

